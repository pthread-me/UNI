\documentclass[cspaper]{IEEEtran}

% Set the margins of the page.
\usepackage[a4paper, total={6.5in, 9in}]{geometry}

% A bunch of math packages.
\usepackage{amssymb}
\usepackage{amsmath}
\usepackage{amsthm}
\usepackage{amsfonts}
\usepackage{mathtools}

\usepackage{graphicx} 		% Insert images
\usepackage{changepage}
\usepackage{color}				% COLORS!
\usepackage[shortlabels]{enumitem}			% More enumerate types such as \alph*
\usepackage{listings}			% Used for code-blocks in latex.
\usepackage{hyperref}

% Create links when using ref and table of contents.
\hypersetup{colorlinks=true, linkcolor=black}

% Replace the indents for paragraphs with empty lines.
\usepackage[parfill]{parskip}

\usepackage[myheadings]{fancyhdr}
\usepackage{titleref}
\makeatletter
\newcommand*{\currentname}{\TR@currentTitle}
\makeatother

% Number equations with reference to sub sections.
\numberwithin{equation}{subsection}

\definecolor{lightgray}{RGB}{220, 220, 220}

%frame=single,
%basicstyle=\fontsize{10pt}{15pt}\selectfont,
%stepnumber=1,
%backgroundcolor=\color{backcolour},   
%commentstyle=\color{codegreen},
%keywordstyle=\color{magenta},
%numberstyle=\tiny\color{codegray},
%stringstyle=\color{codepurple},
\lstdefinestyle{mystyle}{
	%basicstyle=\ttfamily\footnotesize,
    breakatwhitespace=false,         
    breaklines=true,                 
    captionpos=b,                    
    keepspaces=true,                 
    numbers=left,                    
    numbersep=5pt,                  
    showspaces=false,                
    showstringspaces=false,
    showtabs=false,                  
    tabsize=2,
		literate={<-}{$\leftarrow$}{2} {\\infty}{$\infty$}{1} {\\equiv}{$\equiv$}{1}
		{\\IntegerModm}{$(\mathbb{Z}/m\mathbb{Z})$}{7} {\\in}{$\in$}{1} {\\varphim}{$\varphi(m)$}{5}
		{\\integers}{$\mathcal{Z}$}{1} {\\IntegerModn}{$(\mathbb{Z}/n\mathbb{Z})$}{7},
		backgroundcolor=\color{lightgray},
		frame=single,
		framesep=0pt,
		xleftmargin=20pt,
		framexleftmargin=0pt,
		framexrightmargin=-30pt,
		framextopmargin=2pt,
		framexbottommargin=2pt
}

% Set some style rules for code-blocks
\lstset{style=mystyle}


%increases table padding
\def\arraystretch{1.5}



\title{Title Placeholder}
\author{Anas Alhadi\\B00895875}


\begin{document}

	\maketitle
	
	\newpage
	\fancyhead[L]{\text{B00895875}}
	\fancyhead[R]{\text{Anas Alhadi}}
	\thispagestyle{fancy}


	\par{
		\emph{Abstract} -- Static Context Header Compression (SCHC) is an adaptation layer
		capable of achieving large compression ratios on upper layer protocol headers (ie: IPv6, CoAP) by exploiting the presistent and predictable nature of 
		IoT networks to make use of predefined static compression rules that act as a blueprints for the 
		expected network traffic in which a sender can avoid transmitting the entire packet when a matching blueprint is present and instead sends a pointer to the rule. The 
		afformentioned properities that SCHC builds on however makes it unideal for use in mobile networks in which the metadata we want to compress is variable.
		We look into the prospects of introducing dynamic updating of rules by evaluating the performance of scoring/weight assignment heuristics on
		network traffic to predict rules with improved packet coverage. Our results show that in the presence of reasonable assumptions an overall 
		improvement on the average header size is possible. 
	}

	\section{Introduction}
	\par{
			The recent boom of IoT based networks facilitaed the need for a shared networking layer to enable the seemless 
			communication between various types of devices regardless of their underlying communication medium and protocols. As such IPv6 
			has been widley adopted to play that role \textcolor{red}{cite IoV paper}. Yet, this in itself introduced a new set of challenges, specifically for long distance low-power
			communications (ie LoRaWAN networks) that have physical restrictions on the maximum transmission unit of which the size of IPv6 headers often exceeds. Moreover, the 
	}	
	\section{BACKGROUND AND RELATED WORK}
	\section{METHODOLOGY}
	\subsection{Testbed}
	\subsection{Implementation}
	\subsection{Performance Test}

	\section{Results And Analysis}

	\section{Conclusion}
\end{document}
