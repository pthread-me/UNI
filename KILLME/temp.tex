\documentclass{article}

% Set the margins of the page.
\usepackage[a4paper, total={6.5in, 9in}]{geometry}

% A bunch of math packages.
\usepackage{amssymb}
\usepackage{amsmath}
\usepackage{amsthm}
\usepackage{amsfonts}
\usepackage{mathtools}

\usepackage{changepage}
\usepackage{graphicx} 		% Insert images
\usepackage{color}				% COLORS!
\usepackage[shortlabels]{enumitem}			% More enumerate types such as \alph*
\usepackage{listings}			% Used for code-blocks in latex.
\usepackage{hyperref}

% Create links when using ref and table of contents.
\hypersetup{colorlinks=true, linkcolor=black}

% Replace the indents for paragraphs with empty lines.
\usepackage[parfill]{parskip}

\usepackage[myheadings]{fancyhdr}
\usepackage{titleref}
\makeatletter
\newcommand*{\currentname}{\TR@currentTitle}
\makeatother

% Number equations with reference to sub sections.
\numberwithin{equation}{subsection}

\definecolor{lightgray}{RGB}{200, 200, 200}

% Set some style rules for code-blocks
\lstset{
	literate={<-}{$\leftarrow$}{2} {\\infty}{$\infty$}{1},
	backgroundcolor=\color{lightgray}
,
	framexleftmargin=2pt,
	framexrightmargin=2pt,
	framextopmargin=2pt,
	framexbottommargin=2pt,
	frame=single,
	basicstyle=\fontsize{10pt}{15pt}\selectfont,
	stepnumber=1,
	tabsize=4,
}


%increases table padding
\def\arraystretch{1.5}



\title{\textbf{CSCI 8873 -  3 week overview}}

\author{Anas Alhadi\\B00895875}


\begin{document}

	\maketitle

	\section{Main Goal}
	\par{
		To asses different modes of dynamic rule updating on the SCHC compression scheme. 
		And their impact on device and bandwidth usage.
	}


	\section{Network Structure}
	\par{
		We assume a single connection between a SCHC device and a router/core device. All 
		update decisions and calculations are made at the core and communicated back to the device.
	}

	\vspace{20pt}
	\section{Modes}
	\subsection{Static Rules}
	\subsubsection{Description}
	\par{
		Rules are kept as generic as possible, with fields either being known by both device and core
		or explicitly sent (the only compression actions used are sent/not-sent)
	}

	\subsubsection{Details}
	\par{
		The core maintains a priority queue for some fields in it's existing set of rules.
		When sending/recieving a packet we keep note of the number of times a field takes a specific
		value. Once the counter crosses a specified cutoff value, we add a new rule with that value
		set.\\
		We need to consider rule merging, and different cutoff values before adding new rules. 
		This raises an important issue of having too many rules, in which we will need to include a 
		lifetime for rules.
	}

	\newpage
	\subsection{Map Updating}
	\subsubsection{Description}
	\par{
		No new rules are added, instead, variable fields always use the ``Matching" compression
		action, in which we maintain a map of possible values and only transmit the index of the target value.
	}

	\subsubsection{Details}
	\par{
		A simple way of maintaing the ``map" is by having it be a 2 level fixed size priority queue.
		Target values are ordered in the queue based on the number of times they are encountered. Values
		in the top queue are added to the map, while values in the second level queue remain monitored but not compressed.
		Every time an element is swapped in the top level queue it triggers a rule update (which is communicated to the device).\\
		The top level queue's size is fixed as it dictates the number of bits needed to represent the index of the element in the
		map. We test the mode's performance of different map sizes.
	}

	\vspace{20pt}
	\subsection{Field updating}
	\subsubsection{Description}
	\par{
		For field in which the set of target values follow a certain pattern or slightly
		differes (think of the options field in coap in which we access resources in the same path "dir/file\_1") in this case
		(depending on the encoding) we can avoid the transmission of a set of the most significant bits, and instead only
		transmit the parts the vary.
	}

	\subsubsection{Details}
	\par{
		Similar to the previous mode, we keep a 2 level, fixed size, priority queues holding the most
		used target values. We then find the LCP of the values (so their most significant bits) and only transmit the 
		LSB. (I need to verify that this is possible the openSCHC as i'm not sure if we have bit control on compression)
	}


	\newpage
	\section{Short Term Steps:}
	\subsection{Step 1}
	\par{
		By the end of this week we need to have the network (device and core) setup, and prepare the sample set
		of random coap packets for the emulation.
	}

	\subsection{Step 2}
	\par{
	
		By the end of reading week, I need to have finished implementing mode 1 (Static rules) and start running the emulation on the sample packets.
		We record the number of times a rule change occurs and the size of the packet as a measure of performance
	}

	\subsection{Step 3}
	\par{
		By the end of the week after reading week, I need to have a working implementation of mode 2 (Map Updating)
	}

	\subsection{Future steps}
	\par{
		I still need to verify the feasibility of mode 3. further a combination of mode 1 and 2 in which we make use 
		of the 2 level priority queue with 3 levels where level 1 implements Static rules, level 2 implements Mapping, and
		level 3 is uncompressed.
	}

	

\end{document}
