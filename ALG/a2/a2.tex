\documentclass{article}

% Set the margins of the page.
\usepackage[a4paper, total={6in, 8in}]{geometry}

% A bunch of math packages.
\usepackage{amssymb}
\usepackage{amsmath}
\usepackage{amsthm}
\usepackage{amsfonts}
\usepackage{mathtools}

% Insert images
\usepackage{graphicx}

% COLORS!
\usepackage{color}

% More enumerate types such as \alph*
\usepackage{enumitem}

% Used for code-blocks in latex.
\usepackage{listings}

% Create links when using ref and table of contents.
\usepackage{hyperref}
\hypersetup{colorlinks=true, linkcolor=black}

% Replace the indents for paragraphs with empty lines.
\usepackage[parfill]{parskip}

% Number equations with reference to sub sections.
\numberwithin{equation}{subsection}

\definecolor{lightgray}{RGB}{200, 200, 200}

% Set some style rules for code-blocks
\lstset{
	literate={<-}{$\leftarrow$}{2} {\\infty}{$\infty$}{1},
	backgroundcolor=\color{lightgray}
,
	framexleftmargin=2pt,
	framexrightmargin=2pt,
	framextopmargin=2pt,
	framexbottommargin=2pt,
	frame=single,
	basicstyle=\fontsize{10pt}{15pt}\selectfont,
	stepnumber=1,
	tabsize=4,
}
% Custom Commands:
%\renewcommand{\vspace}{\vspace{5pt}}


\title{\textbf{CSCI-4113\\Assignment 2}}
\author{Anas Alhadi\\B00895875}

\begin{document}

	\maketitle
	\tableofcontents

	\newpage 

	\par{
	 	\textbf{\underline{Note:}} To avoid confusing myself, i redefined $y$ as $w$ 
		so the resulting LP $P$ is:
	}


	\begin{equation}\notag
		Mazimize(3x+4w+2z)
	\end{equation}

	Such That:

	\begin{equation}\label{q1-1}\tag{1}
		4x+w-2z \leq 20
	\end{equation}
	\begin{equation}\tag{2}
		3x - w +2 \ge 1 
	\end{equation}
	\begin{equation}\label{q1-3}\tag{3}
		x-2w+2z = -8
	\end{equation}
	\begin{equation}\tag{4}
		4x-w+3z \leq 21
	\end{equation}

	and
	
	\begin{equation}\notag
		x\ge 0
	\end{equation}
	\begin{equation}\notag
		w \le 0
	\end{equation}

	\newpage
	\section{Question 1}
	\par{Given the LP $P$ that is clearly not in Canonical Form as it has a 
	Minimization Objective Function, Lower Bound and equality Constriants, and negative/unbounded variables.\\We define a 
	sequence of equivalent LPs $P^{(i)}, 0\leq i\leq3$ such that $P^{(3)}=P'$ is an 
	LP equivalent to $P$ and in Canonical form:}


	\vspace{15pt}
	\textbf{\underline{First}}
	\par{
		We need to define an LP $P^{(0)}$ that is equivalent to $P$ but with the a negated
		Objective Function for $P^{(0)}$ to be a maximization LP. 
	}
	\begin{enumerate}
		\item Since $P$ and $P^{(0)}$ have the same variables and constraints, 
			this means that they have the same set of feasible solutions.
		\item Minimizing $f(x)$ is the same as maximizing $f'(x)$, thus $P$ and 
			$P'$ have the same Objective Function value.
	\end{enumerate}

	\par{
		$P^{(0)}$ will have an objective Function:
	}

	\begin{equation}\notag
		Maximize(-3x-4w-2z)
	\end{equation}

	\vspace{15pt}
	\textbf{\underline{Second}}
	\par{
		We now construct a new LP  $P^{(1)}$ that is equivalent to $P^{(0)}$ but with no
		equality constraints.\\ 
		That is, we replace all the equalities with inequalities.
		\begin{enumerate}
			\item Since $a=b$ is clearly equivalent to $a\le b$ \&\& $a \ge b$ 
			\item Thus $P^{(0)}$ and $P^{(1)}$ are equivalent since they have the same 
				set of variables, the same objective function and equivalent constraints.  
		\end{enumerate}
	}

	\par{
		Thus we replace constraint \hyperref[q1-3]{(3)} in $P^{(0)}$ with $\downarrow$ in $P^{(1)}$
	}

	\begin{equation}\label{q1-3.1}\tag{3.1}
		x-2w+2z \leq -8
	\end{equation}
	\begin{equation}\label{q1-3.2}\tag{3.2}
		x-2w+2z \ge -8
	\end{equation}

	\newpage
	\textbf{\underline{Third}}
	\par{
		The next step is to replace all the lower bound inequality constraints 
		in $P^{(1)}$ to upper bound constraints, thus constructing a new LP $P^{(2)}$.
	}

	\begin{enumerate}
		\item An inequality $a\le b$ is equivalent to its negation $-a \ge -b$
		\item Just like in the prev step, we are not changing the Objective function
			and replacing some constraints with equivalent one's thus both $P^{(2)}$ and 
			$P^{(1)}$ have the same Objective Function values and set of feasible solutions and are thus
			equivalent.
	\end{enumerate}

	\par{
		So we replace constraint \hyperref[q1-1]{(1)} and \hyperref[q1-3.2]{(3.2)} respectivley with:
	}

	\begin{equation}\tag{1.1}
		-3x+w-z \leq -1
	\end{equation}
	\begin{equation}\tag{3.2.1}
		-x+2w-2z \leq 8
	\end{equation}

	\par{This gives us an LP $P^{(2)}$}

	
	\vspace{15pt}
	\textbf{\underline{Fourth}}
	\par{
		We now need to add non-negativity constarints to $P^{(2)}$.
	}

	\begin{enumerate}
		\item $x$ already has the constraint so we dont need to modify it.
		\item \par{
			$w$ has a non-positivitey constraint, which is equivalent to:
			\begin{equation}\notag
				-w' \ge 0
			\end{equation}
			Thus we replace every instance of $w$ with $-w'$ (note that $w=w'$)
			and the constarint $w\leq 0$ with $w'\ge 0$.
		}

		\item \par{
				The variable $z$ is unbounded. We can add a non-negativitey constarint
				on it by representing it as the differnce of two non-negative numbers.
				It is clear to see that this holds for any number. So:

				\begin{equation}\notag
					z = z' - z''	
				\end{equation}
				And add the bellow constarints to the LP
				\begin{equation}\notag
					z'\ge0, z''\ge0
				\end{equation}
				We then replace all instances of $z$ with $z'-z''$ in $P^{(2)}$ Thus giving us a new LP $P^{(3)}$.
			}
	\end{enumerate}
	
	Notice that all the LP`s in the sequence {0,1,2,3} are equivalent and since $P$ is equivalent to $P^{(0)}$.
	This means that it is also equivalent to $P^{(3)}$. And since $P^{(3)}$ is a maximization 
	LP whom constraints all define upper bounds, and has non-negativitey constarints 
	on all its variables this means that it is in Canonical form, so $P=P^{(3)} = P'$)(The equality
	here denotes equivalence).

	\newpage
	Then $P'$ is our Canonical LP such that (im reusing equation numbers, a hard reset):

	\begin{equation}\notag
		Maximize(-3x+4w-2z'+2z'')
	\end{equation}
	Such that:
	\begin{equation}\tag{1}
		4x-w-2z'+2z''\leq20	
	\end{equation}
	\begin{equation}\tag{2}
		-3x-w-z'+z''\leq-1	
	\end{equation}
	\begin{equation}\tag{3}
		x+2w+2z'-2z''\leq-8
	\end{equation}
	\begin{equation}\tag{4}
		-x-2w-2z'+2z''\leq8	
	\end{equation}
	\begin{equation}\tag{5}
		4x+w+3z'-3z''\leq21	
	\end{equation}
	\begin{equation}\notag
		x\ge0, w\ge0, z'\ge0, z''\ge0	
	\end{equation}


	\newpage 
	\section{Question 2}
	\par{
		An LP is in standard form if it follows all the restrictions of the Canonical form
		with the changed restriction of having constarints be equality instead of inequalities.
		We do this by adding a slack variable $y_i$ for the $i^{th}$ constraint.\\ 
		So LP equivalent to $P$ and $P'$ but in standard form is $P''$ such that:
	}

	\begin{equation}\notag
		Maximize(-3x+4w'-2z'+2z'' +d)
	\end{equation}
	Such that:
	\begin{equation}\tag{1}
		4x-w-2z'+2z''+y_1=20	
	\end{equation}
	\begin{equation}\tag{2}
		-3x-w-z'+z''+y_2=-1	
	\end{equation}
	\begin{equation}\tag{3}
		x+2w+2z'-2z''+y_3=-8
	\end{equation}
	\begin{equation}\tag{4}
		-x-2w-2z'+2z''+y_4=8	
	\end{equation}
	\begin{equation}\tag{5}
		4x+w+3z'-3z''+y_5=21	
	\end{equation}
	\begin{equation}\notag
		x\ge0, w\ge0, z'\ge0, z''\ge0
	\end{equation}
	\begin{equation}
		y_i\ge0 \text{ where } 1\le i\le5	
	\end{equation}


	\newpage
	\section{Question 3}
	\begin{center}
		\begin{tabular}{|c|c|c|c|c|c|c|c|c|c|}
		\hline
		$b$& 	$y_1$& 	$y_2$& 	$y_3$& 	$y_4$& 	$y_5$& 	$x$& 	$w$& 	$z'$& 	$z''$\\
		\hline
		$20$& 	$1$& 	$$& 	$$& 	$$& 	$$& 	$4$& 	$-1$& 	$-2$& 	$2$\\		
		\hline
		$-1$& 	$$& 	$1$& 	$$& 	$$& 	$$& 	$-3$& 	$-1$& 	$-1$& 	$1$\\
		\hline
		$-8$& 	$$& 	$$& 	$1$& 	$$& 	$$& 	$1$& 	$2$& 	$2$& 	$-2$\\
		\hline
		$8$& 	$$& 	$$& 	$$& 	$1$& 	$$& 	$-1$& 	$-2$& 	$-2$& 	$2$\\ 
		\hline
		$21$& 	$$& 	$$& 	$$& 	$$& 	$1$& 	$4$& 	$1$& 	$3$& 	$-3$\\
		\hline
		$$& 	$$& 	$$& 	$$& 	$$& 	$$& 	$-3$& 	$4$& 	$-2$& 	$2$\\
		\hline
		\end{tabular}
	\end{center}

	\vspace{15pt}
	The basic solutions $\{b_1, b_2, b_3, b_4, b_5\}$ are $\{20, -1, -8, 8, 21\}$
	respectivley. This solution is not feasible since $b_1$ and $b_2$ both have 
	negative values.

	\newpage
	\section{Question 4}
	\par{
		We modify the Original LP by adding a new variable $s$ to form an Auxiliary LP $Q$:
	}

	\begin{equation}\notag
		Maximize(-s)
	\end{equation}
	Such that:
	\begin{equation}\tag{1}
		4x-w-2z'+2z''+y_1-s=20	
	\end{equation}
	\begin{equation}\tag{2}
		-3x-w-z'+z''+y_2-s=-1	
	\end{equation}
	\begin{equation}\tag{3}
		x+2w+2z'-2z''+y_3-s=-8
	\end{equation}
	\begin{equation}\tag{4}
		-x-2w-2z'+2z''+y_4-s=8	
	\end{equation}
	\begin{equation}\tag{5}
		4x+w+3z'-3z''+y_5-s=21	
	\end{equation}
	\begin{equation}\notag
		x\ge0, w\ge0, z'\ge0, z''\ge0, s\ge0
	\end{equation}
	\begin{equation}\notag
		y_i\ge0 \text{ where } 1\le i\le5	
	\end{equation}

	\vspace{15pt}
	\textbf{\underline{Steps:}}
	\begin{enumerate}
		\item Choose the column whom basic variable corresponds to the lowest
			basic solution value. In our case the lowest value in $b$ is $-8$
			which corresponds to the variable $y_3$
		\item Swap column of $s$ with $y_3$, so $y_3$ leaves the basis and $s$ enters
		\item Looking at row (4), multiple the entries in (4)  by $-1$, Then:
			\begin{enumerate}
				\item (1) + (4)
				\item (2) + (4)
				\item (3) + (4)
				\item (5) + (4)
				\item (6) + (4)
			\end{enumerate}
			Now we have that $Q$ has a feasible solution, so we continue solving the 
			LP until we find an optimal solution.

		\item We choose an arbirtary non-basic variable whom objective function
			value is non-negative, I'm choosing $z''$, we then find the row with 
			the $Min(\frac{b_i}{a_{z'',i}})$.\\ 
			In this case the min is $\frac{b_5}{a_{w,2}} = \frac{7}{3} = 2.333$ 

		\item Swap columns $z''$ and $y_2$ Then perform the basic row operations to restore the 
			identity matrix in the basis.
			\begin{enumerate}
				\item (2) / 3
				\item (1) - 4*(2) 
				\item (3) - 2*(2)
				\item (4) - 4(2)
				\item (5) - (2)
				\item (6) - 4(2)

			\end{enumerate}

		\item Repeate steps 4 and 5, this time we choose the column $x$ so the 
			$Min(\frac{b_i}{a_{x,i}})$ is $\frac{10}{3}\times \frac{3}{5}$ 
			corresponding to column $s$

			\begin{enumerate}
				\item (3) * 3/5
				\item (1) - 25/3(3)
				\item (2) + 4/3(3)
				\item (4) - 10/3(3)
				\item (5) - 10/3(3)
				\item (6) - 4/3(3)
			\end{enumerate}

		\item Again now with $y_3$ as the variable entering the basis, the corresponding 
			variable to leave is $y_4$.

			\begin{enumerate}
				\item (1) - 2(4)  
				\item (2) + 3/5(4)  
				\item (3) + 1/5(4)  
				\item (6) - 3/5(4)  
			\end{enumerate}

		\item Again this time $w$ enters the basis, and $y_1$ leaves. 
			\begin{enumerate}
				\item (2) + (1)
				\item (5) + 2(1)
				\item (6) - 6(1)
			\end{enumerate}


		\item Again, now $y_4$ enters and $y_3$ leaves the basis 
			\begin{enumerate}
				\item (1) + 2(4) 
				\item (2) + 7/5(4) 
				\item (3) - 1/5(4) 
				\item (5) + 4(4) 
				\item (6) -57/5(4) 
			\end{enumerate}

		\item We now recover the original objective function values of $x,w,z',z''$ and 
			perform the following row operations to fix the basis:
			\begin{enumerate}
				\item (6) - 4(1)
				\item (6) - 2(2)
				\item (6) + 3(3)	
			\end{enumerate}

		\item The final LP is: <insert table bellow>
	\end{enumerate}


	\newpage
	\section{Question 5}


\end{document}
