\documentclass{article}

% Set the margins of the page.
\usepackage[a4paper, total={6.5in, 9in}]{geometry}

% A bunch of math packages.
\usepackage{amssymb}
\usepackage{amsmath}
\usepackage{amsthm}
\usepackage{amsfonts}
\usepackage{mathtools}

\usepackage{graphicx} 		% Insert images
\usepackage{changepage}
\usepackage{color}				% COLORS!
\usepackage[shortlabels]{enumitem}			% More enumerate types such as \alph*
\usepackage{listings}			% Used for code-blocks in latex.
\usepackage{hyperref}

% Create links when using ref and table of contents.
\hypersetup{colorlinks=true, linkcolor=black}

% Replace the indents for paragraphs with empty lines.
\usepackage[parfill]{parskip}

\usepackage[myheadings]{fancyhdr}
\usepackage{titleref}
\makeatletter
\newcommand*{\currentname}{\TR@currentTitle}
\makeatother

% Number equations with reference to sub sections.
\numberwithin{equation}{subsection}

\definecolor{lightgray}{RGB}{220, 220, 220}

%frame=single,
%basicstyle=\fontsize{10pt}{15pt}\selectfont,
%stepnumber=1,
%backgroundcolor=\color{backcolour},   
%commentstyle=\color{codegreen},
%keywordstyle=\color{magenta},
%numberstyle=\tiny\color{codegray},
%stringstyle=\color{codepurple},
\lstdefinestyle{mystyle}{
	basicstyle=\ttfamily\footnotesize,
    breakatwhitespace=false,         
    breaklines=true,                 
    captionpos=b,                    
    keepspaces=true,                 
    numbers=left,                    
    numbersep=5pt,                  
    showspaces=false,                
    showstringspaces=false,
    showtabs=false,                  
    tabsize=2,
		literate={<-}{$\leftarrow$}{2} {\\infty}{$\infty$}{1} {\\equiv}{$\equiv$}{1}
		{\\IntegerModm}{$(\mathbb{Z}/m\mathbb{Z})$}{7} {\\in}{$\in$}{1} {\\varphim}{$\varphi(m)$}{5}
		{\\integers}{$\mathcal{Z}$}{1} {\\IntegerModn}{$(\mathbb{Z}/n\mathbb{Z})$}{7},
		backgroundcolor=\color{lightgray},
		frame=single,
		framesep=0pt,
		xleftmargin=20pt,
		framexleftmargin=0pt,
		framexrightmargin=-30pt,
		framextopmargin=2pt,
		framexbottommargin=2pt
}

% Set some style rules for code-blocks
\lstset{style=mystyle}


%increases table padding
\def\arraystretch{1.5}



\title{\textbf{CSCI-4116\\Assignment 9}}
\author{Anas Alhadi\\B00895875}


\begin{document}

	\maketitle

	\vspace{20pt}
	\par{Note: all code was written to run on Sage Cell Server (sagecell.sagemath.org), with the language option set to Sage}

	\hrulefill

	\vspace{25pt}
	\section*{Question 1}
	\subsection*{Part A}

\vspace{10pt}
\textbf{Code:}

\begin{lstlisting}
	R = IntegerModRing(3511)
	_,x,y = xgcd(65537,3511)

	print("(x,y) =", (x,y))
\end{lstlisting}

\textbf{Output:}
\begin{lstlisting}
	(x,y) = (-1405, 26226)
\end{lstlisting}


\vspace{25pt}
\subsection*{Part B}

\vspace{10pt}
\textbf{Code:}
\begin{lstlisting}
	p, m = 1234567, 100000		# defining the modulo to acquire the last 5 digits
	R = IntegerModRing(m)			#	\IntegerModm	
	a = R(3)									#	3 \in \IntegerModm

	n = euler_phi(m)					# \varphim
	N = IntegerModRing(n)			#	(\integers/\varphim\integers)
	p_mod_n = N(p)						#	calculating the congruence: p_mod_n \equiv p (mod\varphim)

	a^p_mod_n									# 3^\varphim (mod m)
\end{lstlisting}

\newpage
	\fancyhead[L]{\text{B00895875}}
	\fancyhead[R]{\text{Anas Alhadi}}
	\thispagestyle{fancy}




\textbf{Output:}

\begin{lstlisting}
	40587	
\end{lstlisting}


\vspace{25pt}

\subsection*{Part C}

\vspace{10pt}
\textbf{Code:}

\begin{lstlisting}
	a, r, m = 314, 271, 11111				#	For the form a*x \equiv r (mod m)
	R = IntegerModRing(m)

	if gcd(a,m) == 1:
		a_inv = inverse_mod(a, m)
		x = r*a_inv										#	We multiply both sides by the inverse
		x = R(x)											# reducing x to modulo m
		print(x)
	else:
		print("No unique solution")
\end{lstlisting}

\textbf{Output:}
\begin{lstlisting}
	10298
\end{lstlisting}


\vspace{25pt}
\subsection*{Part D}
We first check if a solution exists. Given the form:
\begin{equation}\notag
	ax + by = d	
\end{equation}

A solution exists if and only if $gcd(a,b) | d$. By corollary 5.1 (in the notes)

In our case the equation is:
\begin{equation}\notag
	216x + 606y = 66	
\end{equation}
And 
\begin{equation}\notag
	gcd(216, 606) = 6, \qquad 6|66	
\end{equation}


We now reduce the equation by the gcd (divide it), So:
\begin{equation}\notag
	36x + 101y = 11	
\end{equation}

Such that the $gcd(a,b) = 1$ and we can use the Extended Euclidean Algorithm to find the multiplicative inverse of 
36 in modulo 101. I do this using the bellow code in Sage:

\textbf{Code:}
\begin{lstlisting}
	inverse_mod(36, 101)
\end{lstlisting}
\textbf{Output}
\begin{lstlisting}
	87
\end{lstlisting}

\newpage
	\fancyhead[L]{\text{B00895875}}
	\fancyhead[R]{\text{Anas Alhadi}}
	\thispagestyle{fancy}


So $x=87$ is a solution for $36x+101y=1$ But we want to solve for $11$. Thus we multiply both sides by $11$ (I disregard
the value of y since it disappears when we apply mod 101):
\begin{equation}\notag
	x = 87*11 \equiv 48 (\textrm{mod}\ 101)	
\end{equation}
So:
\begin{equation}\notag
	36(48) + 101y = 11	
\end{equation}

And we have it that $x=48$ is a solution to the congruence $216x \equiv 66(\textrm{mod} \ 606)$


We acuired this solution by restricting the modulo to 101 from 606 (which makes it so that the congruence is solvable).
To extend the modulo back to 606 and find the remaining possible values of x we add multiples of 101 to x.
\begin{equation}\notag
	x = \{48+101n, \quad n = \{0,1,2,3,4,5\}\}	
\end{equation}

So the solution is:
\begin{equation}\notag
	x = \{48, 149, 250, 351, 452, 553 \}	
\end{equation}

\newpage
	\fancyhead[L]{\text{B00895875}}
	\fancyhead[R]{\text{Anas Alhadi}}
	\thispagestyle{fancy}


\section*{Question 2}

\subsection*{Part A}
Given: \ $n=899$ \ $e=11$ \ $c=468$. We compute $\varphi(n) = 840$

We first check the $gcd(e, \varphi(n)) = gcd(11, 840) = 1$ So the key is valid.
To find the decryption key $d$ we need to find the mutiplicative inverse of $e$ in $\mathbb{Z}/\varphi(n)\mathbb{Z}$. 


Using the Extended Euclidean Algorithm, we get that $d = 611$ 

It follows then that the plaintext $m$ is:

\begin{equation}\notag
	m = c^d (\textrm{mod} \ n)		
\end{equation}
\begin{equation}\notag
	m = 468^{611} (\textrm{mod}\ 899)
\end{equation}

\begin{equation}\notag
	m = 13	
\end{equation}

The decryption function in Sage is:

\textbf{Code:}
\begin{lstlisting}
	def decrypt(n,e,c):
		R = IntegerModRing(n)
		c = R(c)									# redefing the cipher text so that: c \in \IntegerModn 
		
		phi_n = euler_phi(n)
		d = inverse_mod(e, phi_n)

		m = c^d
		m = R(m)									#	Same as for c, this is equivalent to: m (mod n)
		
		return m
\end{lstlisting}

Calling the function:
\begin{lstlisting}
	decrypt(899,11,468)
\end{lstlisting}  
returns the same value as before. $m=13$
\vspace{25pt}

\subsection*{Part B}
We repeat the exact same steps, this time calling:\footnote{In both parts a and b, we easily check by raising $m^e (\textrm{mod} \ n)$
which returned $c$ so the function works correctly for the given inputs}
\begin{lstlisting}
	decrypt(11413,7467,5859)
\end{lstlisting}
We get $m=1415$


\newpage
	\fancyhead[L]{\text{B00895875}}
	\fancyhead[R]{\text{Anas Alhadi}}
	\thispagestyle{fancy}

\section*{Question 3}
Given: \ $n=642401$ \ , \ $516107^2 \equiv (\textrm{mod}\ n)$ \ and \ $187722^2 \equiv 2^2 \times 7 (\textrm{mod}\ n)$

\vspace{10pt}
We make use of the \textbf{Basic Princliple} mentioned in appendix A:
\begin{adjustwidth}{2em}{2em}	
If $x^2 \equiv y^2 (\textrm{mod}\ n)$ and $x \not \equiv \pm y (\textrm{mod}\ n)$. Then $gcd(x-y, n) | n$ and is non-trivial (niether $1$ nor $n$) 
\end{adjustwidth}

\vspace{10pt}
Observe that:
\begin{equation}\notag
	(516107)^2 \times (187722)^2 \equiv 2^2 \times 7^2 (\textrm{mod} \ 642401)	
\end{equation}

Since the Ring $(\mathbb{Z}/m\mathbb{Z}, + , \cdot)$ is commutative, we can write it as:
\begin{equation}\notag
	(516107 \times 187722)^2 \equiv (14)^2 (\textrm{mod}\ 642401)	
\end{equation}
\begin{equation}\notag
	(96884638254)^2 \equiv (14)^2 (\textrm{mod}\ 642401)	
\end{equation}


We can confirm using Sage that:

\begin{tabular}{c c c c c c c}
	$95884638254$	&	$\equiv$	&	$289038$	&	$\not \equiv$	&	$14$	&\\
								&						&						&	$\not \equiv$	&	$-14$	& $\equiv$ $642387$
\end{tabular}


So the $gcd(x-y, n)$ is a non-trivial factor:

\begin{equation}\notag
	gcd(96884638254-14, 642401) = 1129	
\end{equation}

And $n$ can be factored as:
\begin{equation}\notag
	n = 642401 = 1129 \times 569	
\end{equation}

\newpage
	\fancyhead[L]{\text{B00895875}}
	\fancyhead[R]{\text{Anas Alhadi}}
	\thispagestyle{fancy}


	\section*{Question 4}

	Given: \ $n=537069139875071$ \ $x=85975324443166$ \ $y=462436106261$

	We first check that $x \not \equiv \pm y (\textrm{mod}\ n)$. Which is trivial since 
	both x and y $<$ n. Then since $x\not = y$ and $x\not = -y+n$, we know that the condition holds.

	Now we can use the ``gcd" function in Sage math to find the factors. By $gcd(x-y, n)$
	We get the factor 9876469. so:

\begin{equation}\notag
	n = 537069139875071 = 9876469 \times 54378659	
\end{equation}

\newpage
	\fancyhead[L]{\text{B00895875}}
	\fancyhead[R]{\text{Anas Alhadi}}
	\thispagestyle{fancy}


\section*{Question 5}
We first implement the Fermat Factorization method which gives us both $q$ and $p$. Then
use them to obtain $\varphi(n)$ to find the inverse.

\textbf{Code:}
\begin{lstlisting}
	n = 152415787501905985701881832150835089037858868621211004433
	R = IntegerModRing(n)

	e = 9007
	c = R(141077461765569500241199505617854673388398574333341423525)


	def factor(n):
		for i in range(1, 1000000):
			cur = n + pow(i, 2)
			if cur.is_square():
				return (cur.sqrt() + i, cur.sqrt() - i)


	p,q = factor(n)
	phi_n = (p-1)*(q-1)
	d = inverse_mod(e, phi_n)

	m = c^d
	print(m)
\end{lstlisting}

\textbf{Output:}
\begin{lstlisting}
	2008091900142113020518002301190014152000190503211805
\end{lstlisting}
\end{document}
