\documentclass{article}

% Set the margins of the page.
\usepackage[a4paper, total={6.5in, 9in}]{geometry}

% A bunch of math packages.
\usepackage{amssymb}
\usepackage{amsmath}
\usepackage{amsthm}
\usepackage{amsfonts}
\usepackage{mathtools}

\usepackage{changepage}
\usepackage{graphicx} 		% Insert images
\usepackage{color}				% COLORS!
\usepackage[shortlabels]{enumitem}			% More enumerate types such as \alph*
\usepackage{listings}			% Used for code-blocks in latex.
\usepackage{hyperref}

% Create links when using ref and table of contents.
\hypersetup{colorlinks=true, linkcolor=black}

% Replace the indents for paragraphs with empty lines.
\usepackage[parfill]{parskip}

\usepackage[myheadings]{fancyhdr}
\usepackage{titleref}
\makeatletter
\newcommand*{\currentname}{\TR@currentTitle}
\makeatother

% Number equations with reference to sub sections.
\numberwithin{equation}{subsection}

\definecolor{lightgray}{RGB}{200, 200, 200}

% Set some style rules for code-blocks
\lstset{
	literate={<-}{$\leftarrow$}{2} {\\infty}{$\infty$}{1},
	backgroundcolor=\color{lightgray}
,
	framexleftmargin=2pt,
	framexrightmargin=2pt,
	framextopmargin=2pt,
	framexbottommargin=2pt,
	frame=single,
	basicstyle=\fontsize{10pt}{15pt}\selectfont,
	stepnumber=1,
	tabsize=4,
}


%increases table padding
\def\arraystretch{1.5}



\title{\textbf{CSCI-4116\\Assignment 5}}
\author{Anas Alhadi\\B00895875}


\begin{document}

	\maketitle

	

	\section*{Question 1}
	Let $a=15$ and $b=18$, so $gcd(15,18) = 3$. 
	
	\vspace{10pt}
	Now $\varphi(15\times18)=\varphi(270)$: 

	\begin{tabular}{c c c c}
		$\varphi(270)$	&	$=$	&	$270 \times \prod_{p|270}(1-\frac{1}{p})$	&	\quad where $p \in \mathbb{P}$ \\
										&	$=$	&	$270 \times (1-\frac{1}{2}) \times (1- \frac{1}{3}) \times (1-\frac{1}{5})$	&	\\
										&	$=$	&	$270 \times \frac{1}{2} \times \frac{2}{3} \times \frac{4}{5}$	& \\
										& $=$	&	$72$
	\end{tabular}

	\vspace{15pt}
	While $\varphi(15) \times \varphi(18)$

	\begin{tabular}{c c c}
		$\varphi(15)$	&	$=$	&	$15 \times \frac{2}{3} \times \frac{3}{5}$ \\
									&	$=$	&	$8$ \\
									&			&			\\
		$\varphi(18)$	&	$=$	&	$18 \times \frac{1}{2} \times \frac{2}{3}$ \\
									&	$=$	&	$6$

	\end{tabular}

	\vspace{15pt}
	And $72 \not = 48$
	
	\newpage
	\fancyhead[L]{\text{B00895875}}
	\fancyhead[R]{\text{Anas Alhadi}}
	\thispagestyle{fancy}


	\section*{Question 2}
	\begin{tabular}{c c c}
		$S$			&	$=$	&	$\{HH, HT, TH, TT\}$ \\	
		$p(HH)$	&	$=$	&	$\frac{1}{4}$ \\
		$p(HT)$	&	$=$	&	$\frac{1}{4}$ \\
		$p(TH)$	&	$=$	&	$\frac{1}{4}$ \\
		$p(TT)$	&	$=$	&	$\frac{1}{4}$ \\
	\end{tabular}

	The event of flipping at least 1 tail, is the subset $e = \{HT, TH, TT\}$. Which is equivelent
	to the event of not getting 2 Heads. So $e =S \setminus \{HH\}$
	
	\begin{tabular}{c c c}
		$p(e)$	&	$=$	&	$p(S) - p(HH)$ \\
						&	$=$	&	$1-\frac{1}{4}$ \\
						&	$=$	&	$\frac{3}{4}$
	\end{tabular}

	\vspace{30pt}
	\section*{Question 3}
	Let $R = \{1,2,3,4,5,6\}$. Then the sample space of rolling 2 dice is $S = R^{2}$. 
	
	\vspace{5pt}
	The question describes 2 events in $S$:

	\begin{tabular}{c c c}
		$e_1$	&	$=$	&	$\{(a,b)\in S, \textrm{where}\ a \not= b\}$ \\
		$e_2$	&	$=$	&	$\{(a,b)\in S, \textrm{where}\ 2|(a+b)\}$
	\end{tabular}

	And asks for the probability that $e_1$ occurs given $e_2$. So $p(e_1|e_2)$, which
	is equivelent to the probability of event $e_1$ occuring when the sample space is restricted to the set $e_2$

	Now observe that $e_2$ is the event in which $a$ and $b$ have the same parity. Thus:
	
	\begin{tabular}{c c c c}
		$e_2$	&	$=$	&	$\{(1,1), (1,3), (1,5),$ 	\\
					& 		&		$(2,2), (2,4), (2,6),$	\\ 
					&			&		$(3,1), (3,3), (3,5),$	\\
					&			&		$(4,2), (4,4), (4,6),$	\\
					&			&		$(5,1), (5,3), (5,5),$	\\
					&			&		$(6,2), (6,4), (6,6)\}$	\\
	\end{tabular}

	We know that $S$ follows a uniform distribution, so the probabilty of each elementary event is $\frac{1}{|S|}$. Since $e_2$ is a
	subset of $S$ this means that the elements in $e_2$ are also uniformally distributed, that is the probability
	of each elementary event in $e_2$ is $\frac{1}{|e_2|}$

	Finally, the probablity of $a\not=b$ in $e_2$ is equal to $1-(\textrm{probability that }a=b)$.

	So:

	\begin{tabular}{c c c}	
		$p(e_1 | e_2)$ &$=$& $1-\frac{6}{|e_2|}$ \\
									&$=$& $1-\frac{6}{18}$		\\
									&$=$&	$\frac{2}{3}$
	\end{tabular}


	\newpage
	\fancyhead[L]{\text{B00895875}}
	\fancyhead[R]{\text{Anas Alhadi}}
	\thispagestyle{fancy}

	\section*{Question 4}
	\subsection*{Part a)}
	We know that the probability $q$ of no two people having the same birthday is:
	
	\begin{tabular}{c c c c}
		$q$	&	$\le$	& $exp(- \frac{k(k-1)}{2n})$
	\end{tabular}

	The probability $p$ of two people having the same birthday is then $p=1-q$. In our case we solve for $p\ge0.9$ and $n=365$

	\vspace{10pt}
	So:

	\begin{equation}\notag
		q	\le	exp(\frac{-k^2+k}{720}) \le 0.1
	\end{equation}
	\begin{equation}\notag
		ln(0.1)\ge \frac{-k^2+k}{720}
	\end{equation}

	\vspace{10pt}
	Rearranging the inequality:

	\begin{equation}\notag
		k^2 - k + 720(ln(0.1)) \ge 0	
	\end{equation}

	Solving for $k$:

	\begin{equation}\notag
		k = \frac{1 \pm \sqrt{1-4\times(720\times ln(0.1))}}{2}	
	\end{equation}

	Giving the value: $k=41.2199...$\\
	Since we cant have fractional values of $k$ we have it that $k=42$

	\vspace{5pt}
	So the number of people needed, $k$, such that the probability of at least 2 having the same birthday is $p\ge\frac{9}{10}$ is $k\ge42$

	\newpage
	\fancyhead[L]{\text{B00895875}}
	\fancyhead[R]{\text{Anas Alhadi}}
	\thispagestyle{fancy}
	
	\subsection*{Part B)}
	Since the PIN cannot start with $0$ we have it that there are $9\times 10^3$ possible 4 digit combinations. So $n=9\times10^3$.

	And we want the probability of at least 2 people having the same PIN to be, $p\ge0.5$ (so $q\le 0.5$).

	We now repeat the same steps in part A). So:

	\begin{equation}\notag
		q \le exp(\frac{-k^2+k}{18\times 10^3})	\le 0.5
	\end{equation}
	\begin{equation}\notag
		ln(0.5) \ge \frac{-k^2+k}{18\times 10^3}
	\end{equation}

	Rearranging:

	\begin{equation}\notag
		k^2-k+ln(0.5) \times 18\times  10^3 \ge 0
	\end{equation}

	Solving for $k$:

	\begin{equation}\notag
		k = \frac{1 \pm \sqrt{1-4(ln(0.5)\times18\times10^3)}}{2}
	\end{equation}

	Giving the value: $k=112.2...$\\
	So there must be $k\ge113$ people to have the probability of at least 2 sharing the same PIN be $p\ge0.5$

	\newpage
	\fancyhead[L]{\text{B00895875}}
	\fancyhead[R]{\text{Anas Alhadi}}
	\thispagestyle{fancy}

	\section*{Question 5}

	First we define the cryptosystem:

	\begin{tabular}{c c c}
		$\mathcal{P}$	&	$=$	&	$\mathbb{Z}_{26}$ \\
		$\mathcal{C}$	&	$=$	&	$\mathbb{Z}_{26}$\\
		$\mathcal{K}$	&	$=$	&	$\mathbb{Z}_{26}$\\
	\end{tabular}

	\vspace{20pt}
	\subsubsection*{Using the definition of perfect secrecy}
	\begin{equation}\notag
		p(w|c) = p(w), \quad w\in \mathcal{P}\ \textrm{and}\  c\in \mathcal{C}
	\end{equation}
	
	Then by Bayes Theorem:

	\begin{equation}\notag
		p(w|c) = \frac{p(w) p(c|w)}{p(c)}
	\end{equation}

	Observe that:
	\begin{enumerate}
		\item Only one key in the Key space has a $p(k)>0$, which is $k=3$
		\vspace{5pt}
		\item We know that the 
			\begin{equation}\notag
				p(c) = \sum p(w)\times p(k_{w,c}) \quad \forall w\in \mathcal{P}, \ \textrm{and}\ k_{w,c} = \{k\in\mathcal{K} \ |\ E_k(w)\mapsto c\}
			\end{equation}
			Then since there is only one key, we have it that the probability of the ciphertext being $c$\\ 
			is equal to the probability $p(w_c)$ where $E_k(w_c) = c$ 
			\vspace{5pt}
	\item $p(c|w)$ asks for the probability that the encryption of the plaintext $w$ results in $c$, so the probability that $E_k(w) \mapsto c$
				. Given that there is only one key, we know that $w$ will either always be mapped to $c$ or never, so $p(c|w) = 1$ or $0$
	\end{enumerate}

	\vspace{10pt}
	It follows then that:
	
	\begin{equation}\notag
		p(w|c) = \frac{p(w_c) \times 1}{p(w_c)} \quad OR \quad \frac{p(w)\times 0}{p(w_c)}
	\end{equation}
		\begin{equation}\notag
			p(w|c) =	1 \quad OR \quad 0
	\end{equation}
	
	Thus we do not have perfect secrecy.

	\vspace{20pt}
	\subsubsection*{Using Shannon's Theorem}
	Shannon's Theorem requires that $\mathcal{K}$ follows a uniform distribution. Notice however that in caesar cipher
	only one key has a non-zero probability. So keys are not uniformly distributed, violating the requiement. Thus
	the cryptosystem does not have perfect secrecy
\end{document}
