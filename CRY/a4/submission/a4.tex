\documentclass{article}

% Set the margins of the page.
\usepackage[a4paper, total={6.5in, 9in}]{geometry}

% A bunch of math packages.
\usepackage{amssymb}
\usepackage{amsmath}
\usepackage{amsthm}
\usepackage{amsfonts}
\usepackage{mathtools}

\usepackage{changepage}
\usepackage{graphicx} 		% Insert images
\usepackage{color}				% COLORS!
\usepackage[shortlabels]{enumitem}			% More enumerate types such as \alph*
\usepackage{listings}			% Used for code-blocks in latex.
\usepackage{hyperref}

% Create links when using ref and table of contents.
\hypersetup{colorlinks=true, linkcolor=black}

% Replace the indents for paragraphs with empty lines.
\usepackage[parfill]{parskip}

\usepackage[myheadings]{fancyhdr}
\usepackage{titleref}
\makeatletter
\newcommand*{\currentname}{\TR@currentTitle}
\makeatother

% Number equations with reference to sub sections.
\numberwithin{equation}{subsection}

\definecolor{lightgray}{RGB}{200, 200, 200}

% Set some style rules for code-blocks
\lstset{
	literate={<-}{$\leftarrow$}{2} {\\infty}{$\infty$}{1},
	backgroundcolor=\color{lightgray}
,
	framexleftmargin=2pt,
	framexrightmargin=2pt,
	framextopmargin=2pt,
	framexbottommargin=2pt,
	frame=single,
	basicstyle=\fontsize{10pt}{15pt}\selectfont,
	stepnumber=1,
	tabsize=4,
}


%increases table padding
\def\arraystretch{1.5}



\title{\textbf{CSCI-4116\\Assignment 4}}
\author{Anas Alhadi\\B00895875}


\begin{document}

	\maketitle

	

	\vspace{25pt}
	\section*{Question 1a}

	\vspace{10pt}
	\par{
		Given a matrix $A$, it's inverse $A^{-1} = (det \ A)^{-1} \times adj \ A$
	}

	\par{
		We first check if $A$ is invertible modulo 2, by finding its determinant and verifying
		that it is relativley prime to 2.
	}


	\vspace{10pt}
	\subsection*{det A:}
	\begin{adjustwidth}{2em}{2em}
		\par{
			I am using row reduction and fixing $i=3$. Since $j=2$ and $j=3$ both result
			in $0$ i'll be omitting their calculation step. So:
		}	

		\begin{equation}\notag 
			det \ A = (-1)^{3+1} \times 1 \times det \ A_{3,1}
		\end{equation}

		\begin{equation}\notag
			det \ A = 1 \times (0-1) = -1
		\end{equation}
		
		\par{
			We thus have it that $det\ A = -1$ which is congruent to $1 (\textrm{mod}\ 2)$ and has 
			an inverse $(det\ A )^{-1} = 1$ in modulo 2. So $A$ is invertible
		}
	\end{adjustwidth}

	\newpage
	\fancyhead[L]{\text{B00895875}}
	\fancyhead[R]{\text{Anas Alhadi}}
	\thispagestyle{fancy}

	\subsection*{adj A}
	\begin{adjustwidth}{2em}{2em}
		The adjoint matrix of $A$ is the transpose of the matrix of A's cofactors. So we first
		find the the matrix of cofactors $C$ then transpose it.
		
		\begin{tabular}{c c c c}
			$C_{1,1} =$ & $(-1)^{1+1} \times det \ A_{1,1} =$ & $1\times (0-0) =$  & $0$\\	
			$C_{1,2} =$ & $(-1)^{1+2} \times det \ A_{1,2} =$ & $-1\times (0-0) =$  & $0$\\	
			$C_{1,3} =$ & $(-1)^{1+3} \times det \ A_{1,3} =$ & $1\times (0-1) =$  & $-1$\\	
			$C_{2,1} =$ & $(-1)^{2+1} \times det \ A_{2,1} =$ & $-1\times (0-0) =$  & $0$\\	
			$C_{2,2} =$ & $(-1)^{2+2} \times det \ A_{2,2} =$ & $1\times (0-1) =$  & $-1$\\	
			$C_{2,3} =$ & $(-1)^{2+3} \times det \ A_{2,3} =$ & $-1\times (0-1) =$  & $1$\\	
			$C_{3,1} =$ & $(-1)^{3+1} \times det \ A_{3,1} =$ & $1\times (0-1) =$  & $-1$\\	
			$C_{3,2} =$ & $(-1)^{3+2} \times det \ A_{3,2} =$ & $-1\times (0-1) =$  & $1$\\	
			$C_{3,3} =$ & $(-1)^{3+3} \times det \ A_{3,3} =$ & $1\times (1-1) =$  & $0$\\	
		\end{tabular}

		\vspace{10pt}
		\par{
			Giving us the matrix:
		}
		\begin{equation}\notag
			C = \begin{pmatrix}
						0  & 0  & -1 \\
						0  & -1 &	1	 \\
						-1 & 1	& 0
					\end{pmatrix}	
		\end{equation}

		\vspace{10pt}
		\par{
			And an adjoint matrix:
		}
		\begin{equation}\notag
			adj \ A = C^{T} = \begin{pmatrix}
														0 & 0 & -1 \\
														0 & -1 & 1 \\
														-1 & 1 & 0
												\end{pmatrix}
		\end{equation}
	\end{adjustwidth}

	\vspace{10pt}
	\subsection*{A$^{-1}$}
	\begin{adjustwidth}{2em}{2em}
		\begin{equation}\notag
			A^{-1} = 1 \times adj \ A \ (\textrm{mod}\ 2) = \begin{pmatrix}
																										0 & 0 & 1 \\
																										0 & 1 & 1 \\
																										1 & 1 & 0
																									\end{pmatrix}	
		\end{equation}	
	\end{adjustwidth}

	\newpage
	\fancyhead[L]{\text{B00895875}}
	\fancyhead[R]{\text{Anas Alhadi}}
	\thispagestyle{fancy}

	\section*{Question 1b}
	\par{
		Again I am using row reduction, and fixing $i=1$.
	}

	\begin{tabular}{c c c}
		$(-1)^{1+1} \times 1 \times det \ A_{1,1} =$ & $1 \times (6-1)$ & $= 5$ \\
		$(-1)^{1+2} \times 2 \times det \ A_{1,2} =$ & $-1 \times (4-3)$ & $= -2$ \\
		$(-1)^{1+3} \times 3 \times det \ A_{1,3} =$ & $3 \times (2-9)$ & $= -21$
	\end{tabular}

	\vspace{10pt}
	\begin{equation}\notag
		det\ A = 5 -2 - 21 = -18
	\end{equation}

	\vspace{30pt}
	\section*{Question 2a}
	\par{
		Recall that a matrix $M$ is invirtable in modulo $p$ iff $gcd(det \ M, p) =1$, 
		thus to find the primes where $M$ is not invirtable, we need to find the prime
		factors of the $det \ M$
	}

	\subsection*{det M}
	\begin{adjustwidth}{2em}{2em}
		
	\par{
		I am using column reduction with $j=1$
	}
	
	\begin{tabular}{c c c}
		$(-1)^{1+1} \times 1 \times det \ M_{1,1} = $ & $1 \times (980-350) = $ & $630$ \\  
		$(-1)^{1+2} \times 1 \times det \ M_{1,2} = $ & $-1 \times (392-56) = $ & $-336$ \\  
		$(-1)^{1+3} \times 1 \times det \ M_{1,3} = $ & $1 \times (50-20) = $ & $30$ \\  
	\end{tabular}

	\begin{equation}\notag
		det \ M = 630 - 336 + 30 = 324	
	\end{equation}

	\end{adjustwidth}
	\par{
		The prime factors of $324$ are $2$ and $3$. Thus $M$ is not invertible in $p \in \{2,3\}$.
	}

	\vspace{30pt}
	\section*{Question 2b}
	\par{
		Just like in Question 2a, we need to find the inverse of the det of M in modulo 101. Then
		multiply it with the adjoint matrix of M.
	}

	\subsection*{Inverse of det M}
	\begin{adjustwidth}{2em}{2em}
		\par{
			We can use the Euclidean algorithm to find the inverse. First observe that 
			$324 \equiv 21 (\textrm{mod}\ 101)$, Thus we find the inverse of $21$ in modulo $101$
		}
	\end{adjustwidth}
	
	\newpage
	\fancyhead[L]{\text{B00895875}}
	\fancyhead[R]{\text{Anas Alhadi}}
	\thispagestyle{fancy}


	\subsubsection*{Euclidean Alg Steps:}
	\begin{adjustwidth}{1em}{1em}
			$101 = 21 (4) + (17)$ \\
			$21 = 17(1) + (4)$ \\
			$17 = 4(4) + 1$

			\vspace{10pt}
			\par{
				Now moving backwards:
			}

			$1 = 17 + 4(-1)$ \\
			$4 = 21 + 17(-1)$ \\
			$17 = 101 + 21(-4)$

			\vspace{10pt}
			\par{
				Subsituting the $2^{nd}$ and $3^{rd}$ equations into the first:
			}

			$1 = (101 + 21(-4)) + 21(-4) + (101 + 21(-4)) (-4)$
			$1 = 101(6) + 21(-24)$

			\vspace{10pt}
			\par{
				Thus:
			}
			\begin{equation}\notag
				1 \equiv 21 \times -24 (\textrm{mod} \ 101)
			\end{equation}
			\begin{equation}\notag
				1 \equiv 21 \times 77 (\textrm{mod}\ 101)	
			\end{equation}

			\vspace{10pt}
			\par{
				And the inverse of the determinant of M, $(det\ M)^{-1} = 77$
			}
	\end{adjustwidth}

	\vspace{20pt}
	\subsection*{Adjoint matrix of M}
	\par{
		Again, we now repeat the exact same steps taken in Question 1a to find the cofactor
		matrix of A but this time on M. This gives us the cofactor matrix $C$
	}

	\begin{equation}\notag
		C = \begin{pmatrix}
					630 & -171 & 9 \\
					-336 & 192 & -12 \\
					30 & -21 & 3
				\end{pmatrix}	
	\end{equation}

	\begin{equation}\notag
		adj\ M = C^T = \begin{pmatrix}
										630 & -336 & 30 \\
										-171 & 192 & -21 \\
										9 & -12 & 3
									\end{pmatrix}
	\end{equation}
	
	\newpage
	\fancyhead[L]{\text{B00895875}}
	\fancyhead[R]{\text{Anas Alhadi}}
	\thispagestyle{fancy}

	\subsection*{Inverse of M}
	\begin{center}
		
	\begin{tabular}{c c c}
		$M^{-1} $ &$=$& 	$(det\ M )^{-1} \times adj\ M	\ (\textrm{mod}\ 101)$ \\
							\\
							&$=$	&	$ 77 \times 	\begin{pmatrix} 
										630 & -336 & 30 \\
										-171 & 192 & -21 \\
										9 & -12 & 3
									\end{pmatrix} \ (\textrm{mod}\ 101)$ 
									\\
									\\	
							&$=$& $\begin{pmatrix}
											48510 & -25872 & 2310 \\
											-13167 & 14784 & -1617 \\
											693 & -924 & 231
										\end{pmatrix}\ (\textrm{mod}\ 101)$
										\\	
										\\

							&$\equiv$& $\begin{pmatrix}
														30 & 85 & 88 \\
														64 & 38 & 100 \\
														87 & 86 & 29
													\end{pmatrix}$
	\end{tabular}
	\end{center}

	\vspace{30pt}
	\section*{Question 3}
	
	\begin{center}
		
	\begin{tabular}{c c}	
	$\mathcal{E}:$ & $ (\mathbb{Z}/2\mathbb{Z})^3 \rightarrow  (\mathbb{Z}/2\mathbb{Z})^3$\\
	$s.t$ & $\mathcal{E}(v) \mapsto Av + b (\textrm{mod}\ 2)$
	\end{tabular}	
	\end{center}
	
	Where:

	\begin{equation}\notag
			A = \begin{pmatrix}
				1 & 0 & 0 \\
				0 & 1 & 0 \\
				0 & 0 & 1
		\end{pmatrix}
	\end{equation}
	\begin{equation}\notag
		b = \begin{pmatrix}
			1 \\ 1 \\ 1
		\end{pmatrix}	
	\end{equation}

	\newpage
	\fancyhead[L]{\text{B00895875}}
	\fancyhead[R]{\text{Anas Alhadi}}
	\thispagestyle{fancy}

	\section*{Question 4}
	\begin{equation}\notag
		A = \begin{pmatrix}
			0 & 1 & 0 \\
			1 & 0 & 0 \\
			0 & 0 & 1
		\end{pmatrix}	
	\end{equation}
	\begin{equation}\notag
		b = \begin{pmatrix}
			18 \\ 2 \\ 11
		\end{pmatrix}	
	\end{equation}

	\vspace{30pt}
	\section*{Question 5}
	\par{
		The corresponding key stream is:\\
		$z = 1010011 \ 1010011 \ 1010011$
	}
	
	\par{
		Resulting in:\\
		$\mathcal{E}_k(w) = 0100000 \ 0100010 \ 0000010$
	}
	\end{document}
