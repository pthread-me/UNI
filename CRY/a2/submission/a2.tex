\documentclass{article}

% Set the margins of the page.
\usepackage[a4paper, total={6.5in, 9in}]{geometry}

% A bunch of math packages.
\usepackage{amssymb}
\usepackage{amsmath}
\usepackage{amsthm}
\usepackage{amsfonts}
\usepackage{mathtools}

\usepackage{changepage}
\usepackage{graphicx} 		% Insert images
\usepackage{color}				% COLORS!
\usepackage[shortlabels]{enumitem}			% More enumerate types such as \alph*
\usepackage{listings}			% Used for code-blocks in latex.
\usepackage{hyperref}

% Create links when using ref and table of contents.
\hypersetup{colorlinks=true, linkcolor=black}

% Replace the indents for paragraphs with empty lines.
\usepackage[parfill]{parskip}

\usepackage[myheadings]{fancyhdr}
\usepackage{titleref}
\makeatletter
\newcommand*{\currentname}{\TR@currentTitle}
\makeatother

% Number equations with reference to sub sections.
\numberwithin{equation}{subsection}

\definecolor{lightgray}{RGB}{200, 200, 200}

% Set some style rules for code-blocks
\lstset{
	literate={<-}{$\leftarrow$}{2} {\\infty}{$\infty$}{1},
	backgroundcolor=\color{lightgray}
,
	framexleftmargin=2pt,
	framexrightmargin=2pt,
	framextopmargin=2pt,
	framexbottommargin=2pt,
	frame=single,
	basicstyle=\fontsize{10pt}{15pt}\selectfont,
	stepnumber=1,
	tabsize=4,
}


%increases table padding
\def\arraystretch{1.5}



\title{\textbf{CSCI-4116\\Assignment 2}}
\author{Anas Alhadi\\B00895875}


\begin{document}

	\maketitle

	

	\vspace{25pt}
	\section*{Question 1}
	\subsection*{Part a)}	
	\par{
		We know that a residue class $a+m\mathbb{Z}$ is invertible iff $gcd(a,m)=1$.
		Thus only the classes represented by $\{1,3,5,7,9,11,13,15\}$ are invertible.}
	\par{
		Given a residue class $a+m\mathbb{Z}$ the inverse is a residue class 
		$a^{-1}+m\mathbb{Z}$ such that $a \times a^{-1} \equiv 1(\textrm{mod}\ 16)$.\footnote{I'm using the euclidean algorithm 
		to find $a^{-1}$ but immiting the steps}
		So we have it that:
	}

	\begin{itemize}[]
		\item{\makebox[4em][l]{$1\times1$}}  $\equiv 1(\textrm{mod}\ 16)$	
		\item{\makebox[4em][l]{$3\times11$}}  $\equiv 1(\textrm{mod}\ 16)$	
		\item{\makebox[4em][l]{$5\times13$}}  $\equiv 1(\textrm{mod}\ 16)$	
		\item{\makebox[4em][l]{$7\times7$}}  $\equiv 1(\textrm{mod}\ 16)$	
		\item{\makebox[4em][l]{$9\times9$}}  $\equiv 1(\textrm{mod}\ 16)$	
		\item{\makebox[4em][l]{$15\times15$}}  $\equiv 1(\textrm{mod}\ 16)$	
	\end{itemize}

	\textbf{aside:}
	The format $x \times y \equiv 1(\textrm{mod}\ m)$ just means that
	$y+m\mathbb{Z}$ is the inverse of $x+m\mathbb{Z}$ and vise versa. Since $(\mathbb{Z}/m\mathbb{Z}, \cdot)$ is a commutative monoid
	we know that $a\cdot a^{-1} = a^{-1}\cdot a$.

	\vspace{15pt}
	\subsection*{Part b)}
	\subsubsection*{Group of Units}
	The group of units is the group of all invertible residue classes, and thus, is the 
	group of all residue classes $a+15\mathbb{Z}$ where $gcd(a,15)=1$. Then given the inverses(similar to partA):
	
	\begin{itemize}
		\item{\makebox[4em][l]{$1\times1$}}  $\equiv 1(\textrm{mod}\ 15)$	
		\item{\makebox[4em][l]{$2\times8$}}  $\equiv 1(\textrm{mod}\ 15)$	
		\item{\makebox[4em][l]{$4\times4$}}  $\equiv 1(\textrm{mod}\ 15)$	
		\item{\makebox[4em][l]{$7\times13$}}  $\equiv 1(\textrm{mod}\ 15)$	
		\item{\makebox[4em][l]{$11\times11$}}  $\equiv 1(\textrm{mod}\ 15)$	
		\item{\makebox[4em][l]{$14\times14$}}  $\equiv 1(\textrm{mod}\ 15)$	
	\end{itemize}

	\vspace{1pt}
	We have it that $(\mathbb{Z}/15\mathbb{Z})^* = \{1,2,4,7,8,11,13,14\}$
	
	\vspace{10pt}
	\subsubsection*{Zero Divisors:}
	We know that the zero divisors of $\mathbb{Z}/m\mathbb{Z}$ are the residue
	classes where $1<gcd(a,m)<m$. This means that the zero divisors is the set of all
	residue classes that are not in $(\mathbb{Z}/m\mathbb{Z})^*$ and not the class $0+m\mathbb{Z}$.
	For $m=15$ the zero divisors are:
	
	\begin{itemize}
		\item $3+15\mathbb{Z}$	
		\item $5+15\mathbb{Z}$	
		\item $6+15\mathbb{Z}$	
		\item $9+15\mathbb{Z}$	
		\item $10+15\mathbb{Z}$	
		\item $12+15\mathbb{Z}$	
	\end{itemize}

	\vspace{25pt}
	\section*{Question 2}
	b) is cryptosystem while a) is not.

	\vspace{10pt}
	\textbf{Reason:}
	
	Recall that one properity of a cryptosystem is that for all encryption keys in the keyspace there must exist a decryption key
	such that decrypting the encryption returns the original plain text.\\
	In the context of the provided scheme, the mapping from $x (\textrm{mod}\ m)\mapsto kx (\textrm{mod}\ m)$ can be reversed
	by multiplying $k$ with its inverse in modulo $m$.\\ 
	Since scheme a) puts no restrictions on $k$ this means that all keys in the keyspace must be invertible in $m$, which is only true
	when $m$ is a prime number, and $26$ is clearly not. Thus there exists a key where its inverse, $k^{-1}$, is not uniquely identifiable.
	Therefore there is no gaurentee that $\mathcal{D}_{k^{-1}} (\mathcal{E}_k(p)) = p$, thus the properity is not satisfied. (The restriction in scheme b gaurentees
	that all keys are invertible).

	\vspace{10pt}
	\textbf{Cryptosystem's details: (scheme b)}
	\begin{itemize}
		\item Plaintext Space: $\mathbb{Z}_{26}$
		\item Ciphertext Space: $\mathbb{Z}_{26}$
		\item Key Space: $\{1,3,5,7,9,11,13,15,17,19,21,23,25\}$.\quad \footnote{any value in $(\mathbb{Z}/26\mathbb{Z})^*$ is a valid key for the
			mapping.}
	\end{itemize}	

	\newpage

	% Replace the B00 with \currentname to replace it with the section name for 
	% questions that take up more than 2 pages
	\newpage
	\fancyhead[L]{\text{B00895875}}
	\fancyhead[R]{\text{Anas Alhadi}}
	\thispagestyle{fancy}

	\section*{Question 3 }	
	\begin{enumerate}
		\item $\varphi(2024) = 880$
		\item $\varphi(2025) = 1080$
		\item $\varphi(8958) = 2984$
	\end{enumerate}
	\footnote*{I wrote a C++ program to calculate Question 3 this for me}

	\vspace{30pt}
	\section*{Question 4}
	
	\underline{\textbf{Plain and Cipher Spaces:}}

	The plain and cipher text can be any string over $\Sigma$. Thus the Plain text and Cipher text space
	is the set of all possible strings over the alphabet, denoted as $\Sigma ^*$

	\underline{\text{aside:}}
	\begin{adjustwidth}{1em}{1em}
	The alphabet $\Sigma = \{A..Z\}$ can be mapped to integers $\mathbb{Z}_{26} = \{0..25\}$ in lexigraphical
	order.	
	\end{adjustwidth}
	

	\vspace{20pt}
	\underline{\textbf{Encryption and Decryption Functions}}
	
	The described encryption and decryption procedures can each be represented as the composition of 
	2 functions: 
	\begin{itemize}
		\item Encryption: function $\mathcal{E}'$ that shifts symbols and a function $R$ that reverses a sequence
		\item Decryption: function $R$ that reverses a sequence and a function $\mathcal{D}'$ that shifts symbols 
	\end{itemize}

	\vspace{5pt}
	First, let a string/sequence $X=\{x_i\}^n_{i=1}$ 

	We now define the functions $\mathcal{E}', \mathcal{D}'$ and $R$ as follows :

	\begin{center}
	\begin{tabular}{c c c}	
	$\mathcal{E}'_{k_1, k_2}(X)$: & $x_i+k_2(\textrm{mod}\ 26)$, & $i|2$ \\
																 & $x_i+k_1(\textrm{mod}\ 26)$, & otherwise\\
																 & 														 & \\
		$\mathcal{D}'_{k_1, k_2}(X)$: & $x_i-k_2(\textrm{mod}\ 26)$, & $i|2$ \\
																 & $x_i-k_1(\textrm{mod}\ 26)$, & otherwise\\
																 &&\\
		$R(X)$: 						&			$x_i = x_{n-i+1}$	

	\end{tabular}
	\end{center}

	\vspace{10pt}
	We can now write the Encryption and Decryption functions as:
		\begin{equation}\notag
			\mathcal{E} = \mathcal{E}_{k_1,k_2}' \circ R				
		\end{equation}
		\begin{equation}\notag
			\mathcal{D} = R \circ \mathcal{D}_{k_1, k_2}'				
		\end{equation}
	
	\newpage
	\underline{\textbf{Key Space:}}


	Observe that the symbols and the operations $\mathcal{E}'$ and $\mathcal{D}'$ that we 
	perform on them is just the group $(\mathbb{Z}/m\mathbb{Z}, +)$ with $m=26$. So the inverse of
	any $k (\textrm{mod}\ 26)$ is $-k (\textrm{mod}\ 26)$. This means that any integer value of $k$ is a 
	valid key. Thus the keyspace is $\mathbb{Z}_{26}$.

	\vspace{20pt}
	\underline{\textbf{Conclusion:}}

	Since the procedure has:
	\begin{enumerate}
		\item a Plain text space
		\item a Cipher text space
		\item a Key space
		\item an Encryption Function
		\item a Decryption Function
		\item Satisfies that $\mathcal{D}(\mathcal{E}(p))=p$ (since all keys are invertible)
	\end{enumerate}

	This means that it is a valid cryptosystem as it satisfies all the properities of one.

	\vspace{30pt}
	\section*{Question 5}
	Given the encryption function:
	\begin{equation}\notag
		\mathcal{E}(x) = ax +b (\textrm{mod}\ m) 
	\end{equation}
	And the restriction that $a$ needs to have a multiplicative inverse in $m$, we have it that for:
	\begin{itemize}
		\item \textbf{m=30:}

			$b \in \mathbb{Z}_{30}$ so $b$ can have $30$ different values\\
			$a \in (\mathbb{Z}/30\mathbb{Z})^*$ so $a$ can have $\varphi(30)=8$
			

			Thus there are $30*8=240$ possible keys 
		
		\item \textbf{m=29:}

			$b \in \mathbb{Z}_{29}$ so $b$ can have $29$ different values\\
			$a \in (\mathbb{Z}/29\mathbb{Z})^*$ so $a$ can have $\varphi(29)=28$
			

			Thus there are $29*28 = 812$ possible keys 
			
	\end{itemize}
	

\end{document}
