\documentclass{article}

% Set the margins of the page.
\usepackage[a4paper, total={6.5in, 9in}]{geometry}

% A bunch of math packages.
\usepackage{amssymb}
\usepackage{amsmath}
\usepackage{amsthm}
\usepackage{amsfonts}
\usepackage{mathtools}

\usepackage{changepage}
\usepackage{graphicx} 		% Insert images
\usepackage{color}				% COLORS!
\usepackage[shortlabels]{enumitem}			% More enumerate types such as \alph*
\usepackage{listings}			% Used for code-blocks in latex.
\usepackage{hyperref}

% Create links when using ref and table of contents.
\hypersetup{colorlinks=true, linkcolor=black}

% Replace the indents for paragraphs with empty lines.
\usepackage[parfill]{parskip}

\usepackage[myheadings]{fancyhdr}
\usepackage{titleref}
\makeatletter
\newcommand*{\currentname}{\TR@currentTitle}
\makeatother

% Number equations with reference to sub sections.
\numberwithin{equation}{subsection}

\definecolor{lightgray}{RGB}{200, 200, 200}

% Set some style rules for code-blocks
\lstset{
	literate={<-}{$\leftarrow$}{2} {\\infty}{$\infty$}{1},
	backgroundcolor=\color{lightgray}
,
	framexleftmargin=2pt,
	framexrightmargin=2pt,
	framextopmargin=2pt,
	framexbottommargin=2pt,
	frame=single,
	basicstyle=\fontsize{10pt}{15pt}\selectfont,
	stepnumber=1,
	tabsize=4,
}


%increases table padding
\def\arraystretch{1.5}



\title{\textbf{CSCI-4116\\Assignment 3}}
\author{Anas Alhadi\\B00895875}


\begin{document}

	\maketitle

	

	\vspace{25pt}
	\section*{Question 1}
	The process described in the question can be represented as the composition of 
	the encryption functions of the $2$ affine ciphers, so:

	Let:
	\begin{equation}\notag
		F_1: \mathbb{Z}_m	\rightarrow \mathbb{Z}_m
	\end{equation}
	\begin{equation}\notag
		s.t \quad x \mapsto ax + b (\textrm{mod}\ m), \qquad x,a,b\in \mathbb{Z}_m	
	\end{equation}

	And:
	\begin{equation}\notag
		F_2: \mathbb{Z}_m	\rightarrow \mathbb{Z}_m
	\end{equation}
	\begin{equation}\notag
		s.t \quad x \mapsto cx + d (\textrm{mod}\ m), \qquad x,c,d\in \mathbb{Z}_m	
	\end{equation}

	It follows then that the composition of both encryptions:
	\begin{equation}\notag
		F_2 \circ	F_1	: \mathbb{Z}_m \rightarrow \mathbb{Z}_m
	\end{equation}
	\begin{equation}\notag
		s.t \quad x \mapsto c(a(x)+b) + d (\textrm{mod}\ m), \qquad x,a,b,c,d\in \mathbb{Z}_m	
	\end{equation}

	Which by proposition 2.1 (in the lecture notes) can be written as:
	\begin{equation}\notag
		F_2 \circ	F_1	: \mathbb{Z}_m \rightarrow \mathbb{Z}_m
	\end{equation}
	\begin{equation}\notag
		s.t \quad x \mapsto kx + h(\textrm{mod}\ m), \quad k = ca,\ h = cb+d \ \textrm{and}\ x,k,h \in \mathbb{Z}_m	
	\end{equation}

	\vspace{10pt}
	Assuming that $k$ and $m$ are relativley prime (so $F_2 \circ F_1$ is a valid encryption function and 
	the composition is a cryptosystem) then observe that the cryptoanalysis techniques we used in [sec 2.3.2] of
	the notes still holds. That is the number of possible key combinations for $F_1$, $F_2$ and $F_2 \circ F_1$ is the
	same since they all have the same restrictions on that values of $a,c,k$ and $b,d,h$ in modulo $m$. Further
	the plaintext and ciphertext space is the same for the 3 ciphers. So security is not increased.

	\newpage
	\fancyhead[L]{\text{B00895875}}
	\fancyhead[R]{\text{Anas Alhadi}}
	\thispagestyle{fancy}

	\section*{Question 2}

	\begin{adjustwidth}{2em}{1em}
		\textbf{Aside:}

		Given an integer $x$ we can represent it as a sequence of decimal values:
		\begin{equation}\notag
			(x_n, ...\ ,x_0), \quad n = \lceil log_{10}x \rceil
		\end{equation}
		So $x$ can be written as the sum:
		\begin{equation}\notag
			x = \sum_{i=0}^n x_i \times 10^i
		\end{equation}
	\end{adjustwidth}

	\vspace{10pt}
	\subsection*{Part a)}
	Given that $10 \equiv 1 (\textrm{mod}\ 9)$ then by proposition 2.1 (in the lecture notes). 

	We have it that:
	\begin{equation}\notag
		10x \equiv x (\textrm{mod}\ 9)
	\end{equation}

	Further, observe that (again by prop 2.1):
	\begin{equation}\notag
		10^n \equiv 1 (\textrm{mod}\ 9), \quad \forall n \in \mathbb{Z}, n\ge 0	
	\end{equation}
	It follows then that:
	\begin{equation}\notag
		x = \sum_{i=0}^n x_i \times 10^i \equiv \sum_{i=0}^n x_i (\textrm{mod}\ 9)
	\end{equation}

	That is $x$ is congruent to the sum of all its digits, mod $9$. So $9$ divides $x$
	iff the sum is equal to $0 (\textrm{mod}\ 9)$.

	\vspace{20pt}
	\subsection*{Part b)}
	Given that $10 \equiv -1 (\textrm{mod}\ 11)$ 
	
	We make the observation that the congruence can be written in $2$ different ways
	depending on the parity of the power of $10$. That is:

	\begin{center}
		\begin{tabular}{c	c c}
			$10^i$ & $\equiv -1 (\textrm{mod}\ 11)$ & $i = 2m+1, m\in \mathbb{Z}$\\
						 & $\equiv 1(\textrm{mod}\ 11)$   & Otherwise
		\end{tabular}
	\end{center}

	\vspace{10pt}
	So for an integer $x$ (similar to part a. but now divide the sum based on the parity of $x_i$):
	\begin{equation}\notag
		x = \sum_{i=0}^n x_i \times 10^i
	\end{equation}
	\begin{equation}\notag
		= \ \sum_{i=0}^{\frac{n}{2}} x_{2i}	\times 10^{2i} \ + \ \sum_{i=0}^{\frac{n}{2}} x_{2i+1} \times 10^{2i+1}
	\end{equation}

	\newpage
	\fancyhead[L]{\text{B00895875}}
	\fancyhead[R]{\text{Anas Alhadi}}
	\thispagestyle{fancy}

	Now applying the congrunce relation on the sum:
	\begin{equation}\notag
	 \equiv \sum_{i=0}^{\frac{n}{2}} x_{2i} (\textrm{mod}\ 11) + \sum_{i=0}^{\frac{n}{2}} -x_{2i+1} (\textrm{mod}\ 11)	
	\end{equation}
	\begin{equation}\notag
	 \equiv \sum_{i=0}^{\frac{n}{2}} x_{2i}  + \sum_{i=0}^{\frac{n}{2}} -x_{2i+1}  \quad (\textrm{mod}\ 11)	
	\end{equation}

	That is, $x$ is congruent to the alternating sum $x_0 - x_1 + x_2 - x_3 \ ... \ $ modulo 11. Thus 11 divies
	$x$ iff the alternating sum is equal to $0(\textrm{mod}\ 11)$.

	\vspace{30pt}
	\section*{Question 3}
	Recall that a monoid, is a semi-group that has a neutral element. The concatination of strings
	is by definition assosiative. So $(\Sigma ^*, \circ)$ is a semi-group which contains $\epsilon =$ the empty set, 
	that satisfies the properity:
	\begin{equation}\notag
		\epsilon \circ a = a \circ \epsilon = a, \qquad \forall a \in \Sigma^*	
	\end{equation}

	So $\epsilon$ is a neutral element and $(\Sigma^*, \circ)$ is a monoid.

	\vspace{10pt}
	On the other hand, for a monoid to be a group, it must satisfy the properity that every element
	in it is invertibe, however $\epsilon$ is the only invertible element in $(\Sigma^*, \circ)$ with $\epsilon^{-1} = \epsilon$.
	Thus it is not a group.

	\vspace{30pt}
	\section*{Question 4}
	Let $a$ and $b$ be permutations in $S_5$ such that:
	\begin{equation}\notag
		a = \begin{Bmatrix}
					1 & 2 & 3 & 4 & 5 \\
					2 & 1 & 3 & 4 & 5
				\end{Bmatrix}	
	\end{equation}
	\begin{equation}\notag
		b = \begin{Bmatrix}
					1 & 2 & 3 & 4 & 5 \\
					1 & 5 & 4 & 3 & 2
				\end{Bmatrix}
	\end{equation}

	Now observe that the composition of the 2 permutations:
	\begin{equation}\notag
		a \circ b = \begin{Bmatrix}
			1 & 2 & 3 & 4 & 5\\
			5 & 1 & 4 & 3 & 2
		\end{Bmatrix} 	
	\end{equation}
	\begin{equation}\notag
		b \circ a = \begin{Bmatrix}
			1 & 2 & 3 & 4 & 5\\
			2 & 5 & 4 & 3 & 1
		\end{Bmatrix}	
	\end{equation}

	Thus permuting a sequence using $b$ then $a$ may return different values compared to when applying
	$a$ then $b$. So the group $S_5$ is not commutative.


	\newpage
	\fancyhead[L]{\text{B00895875}}
	\fancyhead[R]{\text{Anas Alhadi}}
	\thispagestyle{fancy}

	\section*{Question 5}
	\subsection*{Part a)}
	$n!$ \quad since we are simply permuting on the indicies of the bits, so the permutations are strictly in $S_n$

	\subsection*{Part b)}
	$n$ \quad We can shift by values $0 \ ... \ n-1$ before we start looping back.

	\subsection*{Part c)}
	The bitwise negation function is an example of such permutation.

	\vspace{10pt}
	That is, let a function
	\begin{equation}\notag
		f: \{0,1\}^n \rightarrow \{0,1\}^n	
	\end{equation}
	\begin{center}
		\begin{tabular}{c c c c}
			s.t & $f \ x_i =$ & $ 1$ & when $x_i = 0$ \\
			 & $f \ x_i =$ & $ 0$ & Otherwise
		\end{tabular}
	\end{center}

	\vspace{10pt}
	Observe that for any sequence $x \in \{0,1\}^n$, flippling the bits will result
	in a unique sequence $y\in \{0,1\}^n$. Further, every element $y$ in the codomain is mapped
	to by exactly one element $x$ in the domain (where $x$ is just $y's$ negation). That is to say
	$f$ is a bijective function that maps $\{0,1\}^n \rightarrow \{0,1\}^n$. Thus $f$ is a permutation.

	However, since $f \ 1000 \mapsto 0111$ this means $f$ is not a valid bit permutation.

	\vspace{10pt}
	\begin{adjustwidth}{2em}{2em}
		\textbf{Aside:}

		Bit permutations only permute/``shuffle" the indicies of symbols in a sequence. Permutations
		are less restrictive as they allow the subsitution of symbols (which as seen above is not always
		interchangable).
	\end{adjustwidth}
\end{document}
