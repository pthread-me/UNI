\documentclass{article}

% Set the margins of the page.
\usepackage[a4paper, total={6.5in, 9in}]{geometry}

% A bunch of math packages.
\usepackage{amssymb}
\usepackage{amsmath}
\usepackage{amsthm}
\usepackage{amsfonts}
\usepackage{mathtools}

\usepackage{graphicx} 							% Insert images
\usepackage{color}									% COLORS!
\usepackage[shortlabels]{enumitem}	% More enumerate types such as \alph*
\usepackage{listings}								% Used for code-blocks in latex.
\usepackage{hyperref}								% Links
\usepackage{changepage} 						% adjustwidth to add page indents
% Create links when using ref and table of contents.
\hypersetup{colorlinks=true, linkcolor=black}

% Replace the indents for paragraphs with empty lines.
\usepackage[parfill]{parskip}

\usepackage[myheadings]{fancyhdr}
\usepackage{titleref}
\makeatletter
\newcommand*{\currentname}{\TR@currentTitle}
\makeatother


% Number equations with reference to sub sections.
\numberwithin{equation}{subsection}

\definecolor{lightgray}{RGB}{200, 200, 200}

% Set some style rules for code-blocks
\lstset{
	literate={<-}{$\leftarrow$}{2} {\\infty}{$\infty$}{1},
	backgroundcolor=\color{lightgray}
,
	framexleftmargin=2pt,
	framexrightmargin=2pt,
	framextopmargin=2pt,
	framexbottommargin=2pt,
	frame=single,
	basicstyle=\fontsize{10pt}{15pt}\selectfont,
	stepnumber=1,
	tabsize=4,
}


%increases table padding
\def\arraystretch{1.5}



\title{\textbf{CSCI-4116\\Assignment 1}}
\author{Anas Alhadi\\B00895875}


\begin{document}

	\maketitle

	\vspace{20pt}
	
	\par{
		I have read and understood the posted assignment instructions\\
		Anas,
	}

	\hrulefill

	\vspace{25pt}
	\section*{Question 1}
	
	\begin{itemize}
		\item \textbf{Key: } 23
		\item \textbf{Plain Text: } HIIAMJULESIWISHYOUAGOODCRYPTOCLASS
	\end{itemize}

	\vspace{25pt}
	\section*{Question 2}
		WEARENOTINTERESTEDINTHEPOSSIBILITIESOFDEFEAT

	\vspace{25pt}
	\section*{Question 3}
	To prove equivalence, we need to show that congruence relations are reflexive,
	symmetric and transitive.

	\vspace{10pt}
	\begin{enumerate}
		\item \textbf{Reflexivity:} \qquad [\textit{by showing that}: $x \equiv x(\textrm{mod}\ \mathbb{Z}),\ \forall x\in \mathbb{Z}$ ]

			\vspace{5pt}
			\par{
				Let $x,m \in \mathbb{Z}$. Observe that:
		
			\begin{equation}\notag
				x-x=0 \quad \textit{AND} \quad 0=m\times0	
			\end{equation}
			\begin{equation}\notag
				\therefore (x-x) = m\times0 \quad \therefore \quad m | (x-x)
			\end{equation}
				Thus, $x\equiv x (\textrm{mod}\ m),\ \forall x,m \in \mathbb{Z}$
			}

	\newpage
		\item \textbf{Symmetry:} \qquad [\textit{by showing that}: $x \equiv y(\textrm{mod}\ m) \implies y\equiv x(\textrm{mod}\ m) 
			, \ \forall x,y,m \in \mathbb{Z}$ ]
		
		\vspace{5pt}
		\par{
			Suppose that $x\equiv y(\textrm{mod}\ m), \quad x,y,m \in \mathbb{Z} $
		}

		\par{
			This means that $m|y-x$ (by the definition of congruences) which can be written as:
		}
		\begin{equation}\notag
			y-x = m\times k, \qquad k\in \mathbb{Z}	
		\end{equation}
		
		Then by negating both side:
		\begin{equation}\notag
			-y+x = m\times k \times -1 = m\times h,\qquad h=-k
		\end{equation}

		It follows then:
		\begin{equation}\notag
			x-y = m\times h, \qquad h\in \mathbb{Z}	
		\end{equation}
		\begin{equation}\notag
			\therefore \quad m|x-y	
		\end{equation}
		\begin{equation}\notag
			\therefore \quad y\equiv x(\textrm{mod}\ m)	
		\end{equation}

		\vspace{15pt}
		\item \textbf{Transitivity:}
		
		\vspace{5pt}
		\par{
			Suppose that:
		}
		\begin{equation}\notag
			x\equiv y(\textrm{mod}\ m) \qquad AND \qquad y\equiv z(\textrm{mod}\ m), \quad x,y,z,m \in \mathbb{Z}
		\end{equation}

		This means that:
		\begin{equation}\notag
			x=y+km \qquad AND \qquad y=z+lm, \quad l,m \in \mathbb{Z}
		\end{equation}

		Subsituting $y$ into $x$'s equation
		\begin{equation}\notag
			x = z + lm + km = z+nm, \quad n=l+k	
		\end{equation}

		Therefore:
		\begin{equation}\notag
			x = z+nm, \quad n\in \mathbb{Z}
		\end{equation}
		\begin{equation}\notag
				\therefore \quad m|x-z \quad \therefore \quad x \equiv z(\textrm{mod}\ m)	
		\end{equation}
	\end{enumerate}
	
	\fancyhead[L]{\text{B00895875}}
	\fancyhead[R]{\text{Anas Alhadi}}
	\thispagestyle{fancy}

	\newpage
	\section*{Question 4}
	Given:
	\begin{equation}\notag
		a \equiv b(\textrm{mod}\ m) \qquad AND \qquad c \equiv d(\textrm{mod}\ m)
	\end{equation}
	
	We have it that:
	\begin{equation}\tag{1}
		a=b+km \qquad AND \qquad c=d+hm, \quad k,h \in \mathbb{Z}
	\end{equation}

	\thispagestyle{fancy}
	\textbf{Propositions:}
	\begin{enumerate}[(a)]
		\setcounter{enumi}{1}
		\item $a+c \equiv b+d (\textrm{mod}\ m)$
			
			\textbf{proof:}
			\begin{adjustwidth}{15pt}{}
				\par{
					Summing the two equations in (1)
				}
				\begin{equation}\notag
					a+c = b + km + d + hm	
				\end{equation}
				\begin{equation}\notag
					(a+c) = (b+d)+lm, \quad l=k+h 	
				\end{equation}
			
				This means that
				\begin{equation}\notag
					m | (b+d) - (a+c)
				\end{equation}
				\begin{equation}\notag
					\therefore \quad (a+c) \equiv (b+d)(\textrm{mod}\ m)
				\end{equation}
				
				Which is equivalent to 
				\begin{equation}\notag
					(a+c) \equiv b + d(\textrm{mod}\ m)
				\end{equation}
			\end{adjustwidth}	
	
		\vspace{10pt}
		\item $a\times c \equiv b\times d (\textrm{mod}\ m)$
			
			\textbf{proof:}
			\begin{adjustwidth}{15pt}{}
				\par{
					Repeating the steps in (b) but this time multiplying the 2 equations in (1)
				}
				\begin{equation}\notag
					a\times c = (b+km)(d+hm)	
				\end{equation}
				\begin{equation}\notag
					ac = bd + m(bh + dk + khm)	
				\end{equation}
				\begin{equation}\notag
					ac = bd + mn, \qquad n = bh + dk + khm
				\end{equation}

				Thus:
				\begin{equation}\notag
					m | \ bd - ac \qquad \therefore \qquad ac \equiv bd (\textrm{mod}\ m)
				\end{equation}
			\end{adjustwidth}
	\end{enumerate}

	\newpage
	\fancyhead[L]{\text{B00895875}}
	\fancyhead[R]{\text{Anas Alhadi}}
	\thispagestyle{fancy}



	\section*{Question 5}
	
	\begin{enumerate}[(a)]
		\item -44 , -27 , -10 , 7 , 24 , 41 
		\item \{0 , 18 , 36 , 3 , 21 , 39 , 6 , 24 , 42 , 9 , 27 , 45 , 12 , 30 , 48 , 15 , 33 \}
		\vspace{10pt}
		\item \par{	
			\begin{adjustwidth}{10pt}{}
				Since $7$ and $10$ are relativley prime, we can use the CRT to solve the system of
				congrences for $x$. Which should give us a congruence $x(\textrm{mod}\ 7*10)$ that 
				is equivalent to the two congruences:

				\begin{enumerate}[(1)]
					\item $x\equiv 2(\textrm{mod}\ 7)$
					\item $x\equiv 3(\textrm{mod}\ 10)$
				\end{enumerate}
				
				We can write \textbf{(2)} as $x = 10k +3,\ k\in \mathbb{Z}$. Then subsituting \textbf{(2)} into \textbf{(1)}

				\begin{equation}\notag
					10k + 3 \equiv 2(\textrm{mod}\ 7)	
				\end{equation}
				\begin{equation}\notag
					10k \equiv -1(\textrm{mod}\ 7)	
				\end{equation}
				\begin{equation}\notag
					10k \equiv 6 (\textrm{mod}\ 7)	
				\end{equation}
				
				We now reduce both sides of the equation by (mod 7). (since 10 is not in the alphabet of modulo 7)
				
				\begin{equation}\notag
					10k (\textrm{mod}\ 7) \equiv 6 (\textrm{mod}\ 7) (\textrm{mod}\ 7)	
				\end{equation}
				\begin{equation}\notag
					3k \equiv 6 (\textrm{mod}\ 7)	
				\end{equation}

				To find $k$ we need to find a multiplicative inverse of 3 in modulo 7. 
				(Such that $3*3^{-1} \equiv 1(\textrm{mod}\ 7))$.\\ Observe that $3*5 \equiv 1(\textrm{mod}\ 7)$.
				Thus $5$ is a multiplicitve inverse \footnote{We can use the Extended Euclidian Algorithm to find
				the inverse (which i did initially but didn't want to write all the steps)}.

				\vspace{10pt}
				Multiplying both sides by $5$:
				\begin{equation}\notag
					k \equiv 30 (\textrm{mod}\ 7) \equiv 2 (\textrm{mod}\ 7)	
				\end{equation}
				So:
				\begin{equation}\notag
					k = 7h + 2, \ h\in \mathbb{Z}	
				\end{equation}

				Finally, subsituting the above equation into \textbf{(2)}
				\begin{equation}\notag
					x = 10k +3	
				\end{equation}
				\begin{equation}\notag
					x = (7h+2)*10 + 3
				\end{equation}
				\begin{equation}\notag
					x = 70h + 23, \ h \in \mathbb{Z}	
				\end{equation}

				Thus:
				\begin{equation}\notag
					x \equiv 23 (\textrm{mod}\ 70)	
				\end{equation}
				\end{adjustwidth}
		}

	\end{enumerate}


	\newpage
	\thispagestyle{fancy}
	\section*{Question 6}

	\textbf{Plain text:}
	\par{
		all legistation, all government, all society is founded upon the principle
		of mutual concession, politeness, comity, courtesy.
	}

	\textbf{Explaination:}
	\par{
		The reason that this cipher is relatively easy to solve is:
		
		\begin{enumerate}
			\item it does not
		obfuscate any of the possibly unique charactaristics of the plain text. We can easily identify
		letters (ex: all ``e" s have a unique encoding that doesnt change regardless
		of the context that the letter occurs in) thus we can use frequency analysis. 
		\item The cipher also 
		maintains the length of the words (white space and punctuations are not encrypted) which reduces
		the search space as each word can be decrypted seperately 		
	\end{enumerate}

	This means that an encrypted section, say, GWW is now restricted to only $28C3$ combinations
	of letters(by point 2), which can be further reduced by only considering 
	combinations where the $2^{nd}$ and $3^{rd}$ letters are the same (by point 1). Finally 
	since only words are encrypted, this reduces the search space to only english words which fit that
	category (assuming no spelling errors)
	}

	\underline{Bonus acknowledgment:} English is not my native language  
\end{document}
