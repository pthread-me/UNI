\documentclass{article}

% Set the margins of the page.
\usepackage[a4paper, total={6.5in, 9in}]{geometry}

% A bunch of math packages.
\usepackage{amssymb}
\usepackage{amsmath}
\usepackage{amsthm}
\usepackage{amsfonts}
\usepackage{mathtools}

\usepackage{graphicx} 		% Insert images
\usepackage{color}				% COLORS!
\usepackage[shortlabels]{enumitem}			% More enumerate types such as \alph*
\usepackage{listings}			% Used for code-blocks in latex.
\usepackage{hyperref}

% Create links when using ref and table of contents.
\hypersetup{colorlinks=true, linkcolor=black}

% Replace the indents for paragraphs with empty lines.
\usepackage[parfill]{parskip}

\usepackage[myheadings]{fancyhdr}
\usepackage{titleref}
\makeatletter
\newcommand*{\currentname}{\TR@currentTitle}
\makeatother

% Number equations with reference to sub sections.
\numberwithin{equation}{subsection}

\definecolor{lightgray}{RGB}{200, 200, 200}

% Set some style rules for code-blocks
\lstset{
	literate={<-}{$\leftarrow$}{2} {\\infty}{$\infty$}{1},
	backgroundcolor=\color{lightgray}
,
	framexleftmargin=2pt,
	framexrightmargin=2pt,
	framextopmargin=2pt,
	framexbottommargin=2pt,
	frame=single,
	basicstyle=\fontsize{10pt}{15pt}\selectfont,
	stepnumber=1,
	tabsize=4,
}


%increases table padding
\def\arraystretch{1.5}



\title{\textbf{CSCI-4116\\Emerg notes :(}}
\author{Anas Alhadi\\B00895875}


\begin{document}

	\maketitle

	\vspace{20pt}
	
	\hrulefill

	\vspace{25pt}
	\section*{Rec}
	\underline{Recall:}

	Let G be an abelian group, $g\in G$ written $<g> = \{g^k | 0 \le k \le e\}, e = ord_G g. finite$
	\vspace{25pt}
	\section*{Lagrangs Theorem}
	\begin{enumerate}
		\item Euler theorm: if gcd(a,m) = 1 then $a^{\varphi(m)} \equiv 1 (\textrm{mod}\ m)$
		\item if m = p, so a prime, then $\varphi (m) = p-1$
		\item So, for any base $a$, then $a^{p-1} \equiv 1 (\textrm{mod}\ m)$. WHich follows immediatly, and is fermat's little theorme
	\end{enumerate}

	\section*{Theorem 5.11}
	Recall the group mod 13, in which we found subgroubs g=2,3,4 to be generators for cyclic groups. In that case
	we found that the order of the group |<g>| divides the order of G. Mainly  order of G=12. and order $<2>$ was 12, 12|12. ord $<3>$ = 4, 4|12

	Corollary of this is that $g^{|G|} =1 $, Proof: notice tehorems 5.5 and 5.11\\
	in this case $a^{\varphi(m) =1}$ is nothing but a special case if the above. in which 
	% Replace the B00 with \currentname to replace it with the section name for 
	% questions that take up more than 2 pages
	\newpage
	\fancyhead[L]{\text{B00895875}}
	\fancyhead[R]{\text{Anas Alhadi}}
	\thispagestyle{fancy}
 a

\end{document}
