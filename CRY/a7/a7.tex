\documentclass{article}

% Set the margins of the page.
\usepackage[a4paper, total={6.5in, 9in}]{geometry}

% A bunch of math packages.
\usepackage{amssymb}
\usepackage{amsmath}
\usepackage{amsthm}
\usepackage{amsfonts}
\usepackage{mathtools}

\usepackage{graphicx} 		% Insert images
\usepackage{changepage}
\usepackage{color}				% COLORS!
\usepackage[shortlabels]{enumitem}			% More enumerate types such as \alph*
\usepackage{listings}			% Used for code-blocks in latex.
\usepackage{hyperref}

% Create links when using ref and table of contents.
\hypersetup{colorlinks=true, linkcolor=black}

% Replace the indents for paragraphs with empty lines.
\usepackage[parfill]{parskip}

\usepackage[myheadings]{fancyhdr}
\usepackage{titleref}
\makeatletter
\newcommand*{\currentname}{\TR@currentTitle}
\makeatother

% Number equations with reference to sub sections.
\numberwithin{equation}{subsection}

\definecolor{lightgray}{RGB}{200, 200, 200}

% Set some style rules for code-blocks
\lstset{
	literate={<-}{$\leftarrow$}{2} {\\infty}{$\infty$}{1},
	backgroundcolor=\color{lightgray}
,
	framexleftmargin=2pt,
	framexrightmargin=2pt,
	framextopmargin=2pt,
	framexbottommargin=2pt,
	frame=single,
	basicstyle=\fontsize{10pt}{15pt}\selectfont,
	stepnumber=1,
	tabsize=4,
}


%increases table padding
\def\arraystretch{1.5}



\title{\textbf{CSCI-4116\\Assignment 7}}
\author{Anas Alhadi\\B00895875}


\begin{document}

	\maketitle

	\vspace{20pt}
	
	\hrulefill

	\vspace{25pt}
	\section*{Question 1}
	Given a polynomial $f \in (\mathbb{Z}/2\mathbb{Z})[x]$ where $deg(f) =5$. The
	possible factors of $f$ are \\polynomials $g, h \in (\mathbb{Z}/2\mathbb{Z})[x]$ such that
	$g$ and $h$ either have degrees 1 and 4, or 2 and 3.
	
	To test for polynomials of degrees 1 and 4, all we need to do is determine if $f$ has a zero of  1 or 0. If
	not then there is no linear polynomial that factors it (aka neither $x+1$ nor $x$ factor it, respectively).

	To test for polynomials of degrees 2 and 3 we divide $f$ by all the possible degree 2 polynomials in 
	$(\mathbb{Z}/2\mathbb{Z})[x]$. Those being: $x^2, x^2+1, x^2+x, x^2+x+1$.

	\vspace{10pt}
	\begin{enumerate}
		\item \textbf{x$^5$+x$^4$+1:}
			\begin{adjustwidth}{1em}{1em}
				\begin{itemize}
					\item \underline{Degrees 1 and 4:}
						\begin{equation}\notag
							(1)^5 + (1)^4 + 1 = 1
						\end{equation}
						\begin{equation}\notag
							(0)^5 + (0)^4 +1 = 1
						\end{equation}
						\par{
							So neither $x+1$ nor $x$ factor it.	
						}
					
					\vspace{15pt}
				\item \underline{Degrees 2 and 3:}
					\begin{center}
						\begin{tabular}{c c c c}
							$x^5+x^4+1$ & $=$ & $(x^2) \times (x^3+x^2)$ 			& $+1$\\
													&	$=$	&	$(x^2+1)\times (x^3+x^2+x+1)$	& $+ x$\\
													&	$=$	&	$(x^2+x)\times (x^3)$					&	$ + 1$\\
													&	$=$	&	$(x^2+x+1) \times (x^3+x+1)$	&	$+0$
						\end{tabular}
					\end{center}		
					\par{
						So $x^2+x+1$ and $x^3+x+1$ are factors of $x^5+x^4+1$ in $(\mathbb{Z}/2\mathbb{Z})[x]$ thus it is \underline{reducible}
					}
				\end{itemize}
			\end{adjustwidth}

	\end{enumerate}
		\newpage
		\fancyhead[L]{\text{B00895875}}
		\fancyhead[R]{\text{Anas Alhadi}}
		\thispagestyle{fancy}

		\begin{enumerate}
			\setcounter{enumi}{1}
			\item \textbf{x$^5$+x$^3$+x$^2$+1:}
				\begin{itemize}
					\item \underline{Degrees 1 and 4:}

						Observe that:
						\begin{equation}\notag
							(1)^5+(1)^3+(1)^2+1 = 0
						\end{equation}
						This means that $x-1 = x+1$ is a factor, thus it is \underline{reducible}.

						\underline{Side note:} we can confirm the degree 4 factor by dividing $\frac{x^5+x^3+x^2+1}{x+1} = x^4+x^3+x+1$ 
				\end{itemize}
			
			\vspace{20pt}
			\item \textbf{x$^5$+x$^2$+1}
				\begin{itemize}
					\item \underline{Degrees 1 and 4:}
						\begin{equation}\notag
							(1)^5 + (1)^2 +1 = 1	
						\end{equation}
						\begin{equation}\notag
							(0)^5 + (0)^2 + 1 = 1	
						\end{equation}

				\item \underline{Degrees 2 and 3:}
				\begin{center}
					\begin{tabular}{c c c c}
						$x^5 + x^2 + 1$	&	$=$	&	$(x^2)\times (x^3+1)$ &	$+1$\\
														&	$=$	&	$(x^2+1)\times (x^3+x+1)$	&	$+x$\\
														&	$=$	&	$(x^2+x)\times (x^3+x^2+x)$	&	$+1$\\
														&	$=$	&	$(x^2+x+1)\times(x^3+x^2)$	&	$+1$
					\end{tabular}

				\end{center}	
				Thus the polynomial is irreducible as it cannot be represented by the multiplication of any two non-zero polynomials $g,h\in (\mathbb{Z}/2\mathbb{Z})[x]$
		\end{itemize}
		\end{enumerate}
	
	\newpage
	\fancyhead[L]{\text{B00895875}}
	\fancyhead[R]{\text{Anas Alhadi}}
	\thispagestyle{fancy}

	\section*{Question 2}
	\subsection*{Part A}	
	Given polynomials $f,g,h \in R[x]$ we say that $g\equiv h(\textrm{mod}\ f)$ iff $f|g-h$. So for this question, we divide $(g-h)$ by
	$f$ in module 2, and if the remainder is $0$ then the congruence holds.

	\vspace{10pt}
	\begin{enumerate}
		\item $x^4 \equiv x+1 (\textrm{mod}\ x^4+x+1)$:
				\begin{equation}\notag
					g - h = x^4 - (x+1) = x^4+x+1			
				\end{equation}
	
			\begin{adjustwidth}{1em}{1em}
			Here $g-h = f$ so it clearly divides it and the congruence holds.
				
			\end{adjustwidth}

			\vspace{15pt}
		\item $x^8 \equiv x^2 +1 (\textrm{mod}\ x^4+x+1)$:
				\begin{equation}\notag
					g - h = x^8 + x^2 + 1				
				\end{equation}
				\begin{adjustwidth}{1em}{1em}
					Dividing $(g-h)$ by $f$ (using long division) we get a quotient:
				\end{adjustwidth}
				\begin{equation}\notag
					\frac{x^8+x^2+1}{x^4+x+1} = x^4 + x + 1			
				\end{equation}
			
				\begin{adjustwidth}{1em}{1em}
					So $g-h= f\times(x^4+x+1) + 0$. Thus $f$ divides it and the congruence holds.	
				\end{adjustwidth}

			\vspace{15pt}
		\item $x^{16} \equiv x(\textrm{mod}\ x^4+x+1)$:
			\begin{adjustwidth}{1em}{1em}
				Similar to the previous cases, we have $g-h = x^{16}+x$ and dividing it by $f$ in modulo 2 results in 
				a quotient: $x^{12}+x^9+x^8+x^6+x^4+x^3+x^2+x$, and a remainder of $0$
			\end{adjustwidth}
	\end{enumerate}


	\subsection*{Part B}
	We know that $x^{16}+x = q \times f$ (where $q$ is the quotient in part A). Dividing both sides by $x$,
	we get that $x^{15}+1 = \frac{q}{x} \times f$. Where $\frac{q}{x}$ is a valid polynomial and is equal to: 
	$x^{11}+x^8+x^7+x^5+x^3+x^2+x+1$.\\
	Thus the congruence holds.

	% Replace the B00 with \currentname to replace it with the section name for 
	% questions that take up more than 2 pages
	\newpage
	\fancyhead[L]{\text{B00895875}}
	\fancyhead[R]{\text{Anas Alhadi}}
	\thispagestyle{fancy}
 	\section*{Question 3}
	To compute the product in the field, we simply multiply the 2 polynomials then reduce modulo $x^5+x^2+1$ (by dividing
	and getting the remainder)

	\begin{center}
		\begin{tabular}{c c c}
			$(x^4+x^3) \times (x^3+x^2+1)$	&	$=$	&	$x^7+x^6+x^4+x^6+x^5+x^3$\\
																			&	$=$	&	$x^7+x^5+x^4+x^3$
		\end{tabular}
	\end{center}

	Then performing the long division: $\frac{x^7+x^5+x^4+x^3}{x^5+x^2+1}$, results in a quotient value $x^2+1$ and a 
	remainder of $x^3+1$.
	
	\vspace{15pt}
	
	So, $(x^4+x^3)\times (x^3+x^2+1) = x^3+1$

	\newpage
	\fancyhead[L]{\text{B00895875}}
	\fancyhead[R]{\text{Anas Alhadi}}
	\thispagestyle{fancy}
 	\section*{Question 4}
	The Rijndael polynomial is: $x^8+x^4+x^3+x+1$.

	The 2 bytes can be represented as the polynomials:
	\begin{itemize}
		\item $00000111$ = $x^2+x+1$
		\item $10101011$ = $x^7+x^5+x^3+x+1$
	\end{itemize}
	
	We now mutiply the 2 bytes and reduce them modulo the standard polynomial:
	\begin{center}
		\begin{tabular}{c c c}
			$(x^2+x+1)\times (x^7+x^5+x^3+x+1)$	&	$=$ & $x^9+x^8+x^6+x^4+1$
		\end{tabular}
	\end{center}

	Dividing the resulting product by $x^8+x^4+x^3+x+1$. Results in a quotient value of: $x+1$ and \\
	a remainder: $x^6+x^5+x^4+x^3+x^2$

	\vspace{10pt}
	Thus the result of the multiplication in $GF(2^8)$ is $x^6+x^5+x^4+x^3+x^2$, which corresponds to the byte:
	$01111100$

	\newpage
	\fancyhead[L]{\text{B00895875}}
	\fancyhead[R]{\text{Anas Alhadi}}
	\thispagestyle{fancy}
 	\section*{Question 5}
	\subsection*{Part A}
	Here we need to show that $x^2$ has no linear factors (so polynomials of degree = 1).
	We repeat the process used in Question 1 to test for degrees 1 and 4, however in this case, that values
	that we subsitute in place of $x$ are $a\in (\mathbb{Z}/3\mathbb{Z}) = \{0,1,2\}$.
	
	\vspace{20pt}
		\begin{equation}\notag
			(0)^2+1 =1	
		\end{equation}
		\begin{equation}\notag
			(1)^2 +1 = 2 		\end{equation}
			\begin{equation}\notag
				(2)^2 +1 = 5 = 2 		
			\end{equation}

		Thus the polynomial has no linear factors, which means that it is irreducible.

		\vspace{25pt}
		\subsection*{Part B}
		The residue classes of the polynomial are $0,1,2,x,x+1,x+2,2x,2x+1,2x+2$. 

		We set $\alpha $ to be the root of the polynomial so $\alpha^2+1 = 0$. Thus $\alpha^2 = -1 = 2$

		\begin{center}
			\begin{tabular}{| c | | c | c | c | c | c | c | c | c | c |}
				\hline
				$+$ 				& $0$ &	$1$	&	$2$	&	$\alpha$	&	$\alpha +1$	& $\alpha +2$	&	$2\alpha$	&	$2\alpha +1$	&	$2\alpha +2$\\ \hline \hline
				$1$					&	$1$	&	$2$	&	$0$	&	$\alpha +1$	&	$\alpha +2$	&	$\alpha$	&	$2\alpha+1$	&	$2\alpha +2$	&	$2\alpha$	\\ \hline	
				$2$					&	$2$	&	$0$	&	$1$	&	$\alpha +2$	&	$\alpha$	&	$\alpha +1$	&	$2\alpha+2$	&	$2\alpha$	&	$2\alpha+1$	\\ \hline	
				$\alpha$		&	$\alpha$	&	$\alpha+1$	&	$\alpha +2$	&	$2\alpha$	&	$\alpha+1$	&	$2\alpha+2$	&	$0$	&	$1$	&	$2$	\\ \hline	
				$\alpha +1$	&	$\alpha+1$	&	$\alpha+2$	&	$\alpha$	&	$2\alpha+1$	&	$2\alpha+2$	&	$2\alpha$	&	$1$	&	$2$	&	$0$	\\ \hline	
				$\alpha +2$	&	$\alpha +2$	&	$\alpha$	&	$\alpha+1$	&	$2\alpha+2$	&	$2\alpha$	&	$2\alpha+1$	&	$2$	&	$0$	&	$1$	\\ \hline	
				$2\alpha$		&	$2\alpha$	&	$2\alpha+1$	&	$2\alpha+2$	&	$0$	&	$1$	&	$2$	&	$\alpha$	&	$\alpha+1$	&	$\alpha+2$	\\ \hline	
				$2\alpha +1$&	$2\alpha+1$	&	$2\alpha+2$	&	$2\alpha$	&	$1$	&	$2$	&	$0$	&	$\alpha+1$	&	$\alpha+2$	&	$\alpha$	\\ \hline
				$2\alpha+2$	&	$2\alpha+2$	&	$2\alpha$	&	$2\alpha +1$	&	$2$	&	$0$	&	$1$	&	$\alpha+2$	&	$\alpha$	&	$\alpha+1$	\\ \hline	
			\end{tabular}
		\end{center}

		\vspace{10pt}
		\begin{center}
			\begin{tabular}{| c | | c | c | c | c | c | c | c | c | c |}
				\hline
				$\times$ 				& $0$ &	$1$	&	$2$	&	$\alpha$	&	$\alpha +1$	& $\alpha +2$	&	$2\alpha$	&	$2\alpha +1$	&	$2\alpha +2$\\ \hline \hline
				$0$							&	$0$	&	$0$	&	$0$	&	$0$	&	$0$	&	$0$	&	$0$	&	$0$	&	$0$	\\ \hline	
				$1$							&	$0$	&	$1$	&	$2$	&	$\alpha$	&	$\alpha+1$	&	$\alpha+2$	&	$2\alpha$	&	$2\alpha+1$	&	$2\alpha+2$	\\ \hline	
				$2$							&	$0$	&	$2$	&	$1$	&	$2\alpha$	&	$2\alpha+2$	&	$2\alpha +1$	&	$\alpha$	&	$\alpha +2$	&	$\alpha +1$	\\ \hline	
				$\alpha$				&	$0$	&	$\alpha$	&	$2\alpha$	&	$2$	&	$\alpha +2$	&	$2\alpha+2$	&	$1$	&	$\alpha +1$	&	$2\alpha+1$	\\ \hline	
				$\alpha +1$			&	$0$	&	$\alpha +1$	&	$2\alpha +2$	&	$\alpha +2$	&	$2\alpha$	&	$1$	&	$2\alpha +1$	&	$2$	&	$\alpha$	\\ \hline	
				$\alpha +2$			&	$0$	&	$\alpha+2$	&	$2\alpha +1$	&	$2\alpha +2$	&	$1$	&	$\alpha$	&	$\alpha +1$	&	$2\alpha$	&	$2$	\\ \hline	
				$2\alpha$				&	$0$	&	$2\alpha$	&	$\alpha$	&	$1$	&	$2\alpha +1$	&	$\alpha +1$	&	$2$	&	$2\alpha +2$	&	$\alpha +2$	\\ \hline	
				$2\alpha +1$		&	$0$	&	$2\alpha +1$	&	$\alpha +2$	&	$\alpha +1$	&	$2$	&	$2\alpha$	&	$2\alpha +2$	&	$\alpha$	&	$1$	\\ \hline	
				$2\alpha +2$		&	$0$	&	$2\alpha +2$	&	$\alpha+1$	&	$2\alpha +1$	&	$\alpha$	&	$2$	&	$\alpha+2$	&	$1$	&	$2\alpha$	\\ \hline	
			\end{tabular}	
		\end{center}
\end{document}
